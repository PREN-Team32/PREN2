\subsection{Soll-/ Ist-Zeitvergleich}
\label{sec:SollIstVergleich}
Neben der allgemeinen Projektplanung mittels eines Gantt-Diagrammes, wurde noch ein separater, 
ausführlicher Soll-/ Ist-Zeitvergleich gemacht (siehe Tabelle \ref{tab:SollIstTabelle}). 
Hier sind die Arbeitsschritte der  Projektplanung eins zu eins oder zusammengefasst übernommen. 
Die geplante oder anfänglich geschätzte  Zeit ist der tatsächlich aufgewendeten Zeit 
gegenübergestellt. So sind allfällige Abweichungen von  der Planung gut ersichtlich. Wie in 
der Tabelle \ref{tab:SollIstTabelle} zu sehen ist, wurde die nötige Zeit bei vielen 
Arbeitsschritten anfänglich geringer geschätzt, als sie tatsächlich ausfiel. Dies kann mehrere 
Gründe haben. Zum einen ist die Schätzung zu nennen. Ist diese durch die verantwortliche 
Person zu gering angesetzt worden, schlägt sich das in einer Abweichung nieder. Zum andern 
können auch weitere, unvorhergesehene Arbeiten oder Probleme dazukommen, welche man bei der 
anfänglichen Schätzung noch nicht berücksichtigen konnte. Bei den allgemeinen Projektarbeiten 
konnte durchaus von der ausführlichen Vorarbeit in PREN1 profitiert werden. Viele Dokumentvorlagen 
oder Abläufe konnten übernommen werden und spezifische Aufgaben waren bereits verteilt. Dadurch 
wurde nahezu 20\% weniger Zeit aufgewendet, als anfänglich gedacht. In den einzelnen Disziplinen 
ist die effektiv aufgewendete Arbeitszeit durchgehend höher als geplant. Vieles war Neuland für 
die betreffenden Personen. Von daher war zu erwarten, dass es gewisse Abweichungen gibt. Die 
abschliessenden Arbeiten wie Testen und Parametereinstellung wurden ganz klar unterschätzt. 
Hier stellten sich diverse Einstell- und Abstimmarbeiten als komplizierter dar, als anfangs 
geplant. Dies ist im fast doppelt so grossen Zeitaufwand auch gut ersichtlich. 
\begin{zebralongtable}{p{0.52\textwidth}p{0.1\textwidth}p{0.1\textwidth}p{0.15\textwidth}}
    \caption{Soll-/ Ist-Zeitvergleich in tabellarischer Darstellung}
    \endlastfoot
    \rowcolor{gray}\multicolumn{4}{l}{\textbf{Allgemeine Projektarbeiten}}\\
    \textbf{Aktivität}                & \textbf{Planung} 
                                            & \textbf{Ist} 
                                                  & \textbf{Abweichung}\\
    Input                             & 28  &  28 & 0\\
    Planungssitzungen                 & 49  &  42 & -7\\
    Dokumentation                     & 200 & 150 & -50\\
    Risikomanagement                  & 10  &   5 & -5\\
    Projektplanung                    & 50  &  45 & -5\\
    Plakat/Auftritt                   & 10  &  12 & +2\\
    Präsentation                      & 30  &  28 & -2\\
    \textbf{Gesamt}                   & \textbf{377} 
                                            & \textbf{310} 
                                                  & \textbf{-67}\\
                                      &     &     & \\
    \rowcolor{gray}\multicolumn{4}{l}{\textbf{Arbeiten Maschinentechnik}}\\
    \textbf{Aktivität}                & \textbf{Planung} 
                                            & \textbf{Ist} 
                                                  & \textbf{Abweichung}\\
    Komponenten Tests                 & 25  & 35  & +10\\
    Produktion                        & 80  & 90  & +10\\
    Montage                           & 40  & 40  & 0\\
    Pläne/Zeichnungen/Grafiken        & 10  & 35  & +25\\
    \textbf{Gesamt}                   & \textbf{155} 
                                           & \textbf{200} 
                                                  & \textbf{+45}\\
                                      &     &     & \\
    \rowcolor{gray}\multicolumn{4}{l}{\textbf{Arbeiten Elektrotechnik}}\\
    \textbf{Aktivität}                & \textbf{Planung} 
                                            & \textbf{Ist}
                                                  & \textbf{Abweichung}\\
    Schema                            & 30  & 40  & +10\\
    Print Prototyp                    & 60  & 60  & 0\\
    Print definitiv                   & 15  & 15  & 0\\
    Firmware                          & 40  & 45  & +5\\
    Stepper Board                     & 12  & 8   & -4\\
    DC Motor                          & 3   & 8   & +5\\
    Inbetriebnahme BLDC               & 50  & 30  & -20\\
    Inbetriebnahme                    & 15  & 30  & +15\\
    \textbf{Gesamt}                   & \textbf{225} 
                                            & \textbf{236} 
                                                  & \textbf{+11}\\
                                      &     &     & \\
    \rowcolor{gray}\multicolumn{4}{l}{\textbf{Arbeiten Informatik}}\\
    \textbf{Aktivität}                & \textbf{Planung} 
                                            & \textbf{Ist} 
                                                  & \textbf{Abweichung}\\
    Bluetooth-Connection              & 50  & 40  & -10\\
    Desktop-Applikation               & 20  & 20  & 0\\
    Freedom-Board Kommunikation       & 30  & 35  & +5\\
    Android Zusatzkomponenten         & 12  & 15  & +3\\
    Kamera-Komponente                 & 12  & 8   & -4\\
    Winkelberechnung                  & 6   & 14  & +8\\
    Mergen                            & 10  & 40  & +30\\
    Integrationstest                  & 30  & 30  & 0\\
    Wireless                          & 30  & 20  & -10\\
    \textbf{Gesamt}                   & \textbf{200} 
                                            & \textbf{222} 
                                                  & \textbf{+22}\\
                                      &     &     & \\
    \rowcolor{gray}\multicolumn{4}{l}{\textbf{Gemeinsame Arbeiten am Funktionsmuster}}\\
    \textbf{Aktivität}                & \textbf{Planung} 
                                      & \textbf{Ist} 
                                            & \textbf{Abweichung}\\
    Optimierung/Parametereinstellung  & 85  & 160 & +75\\
    \textbf{Gesamt}                   &  \textbf{85}   
                                            & \textbf{160}  
                                                  & \textbf{+75}\\
    \label{tab:SollIstTabelle}
\end{zebralongtable} 