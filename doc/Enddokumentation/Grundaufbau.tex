\section{Grundaufbau}
	Im kommenden Kapitel wird der Grundaufbau des kompletten Ballwerfers ausführlich erläutert. 
	In diesem Sinne wird auf sämtliche Hard- und Software-Komponenten eingegangen.
    Der Ballwerfer ist so konzipiert, das er aus einem fix stehenden Basismodul besteht, 
    welches in der Mitte des Startbereiches positioniert wird. Die Abwurfeinheit, welche 
    den Ballwurfmechanismus und die Ballzuführung beinhaltet, ist auf dem Basismodul 
    drehend gelagert. Weiter ist auch das Smartphone für die Korberkennung und alle 
    Steuereinheiten auf dem Basismodul angebracht. Das Startsignal wird mittels WLAN 
    von einem externen Notebook übertragen. Im folgenden Bild sieht man die komplette Ballmaschine mit den Hauptkomponenten, Förderband
    Ausrichteinheit mit Brushlessmotor, Beschleunigungsräder und den Netzteilen, siehe 
    Abbildung \ref{abb:Ballmaschine}.
    \begin{figure}[h!]
    	\includegraphics[width=1\textwidth,clip,trim=0mm 0mm 0mm 20mm]
    	{Enddokumentation/Bilder/Geraeteuebersicht_2.jpg}
    	\centering
    	\caption{Aufsicht auf die rechte Seite}
    	\label{abb:Ballmaschine}
    \end{figure}
    
    Der ganze Aufbau des Ballwurfmechanismus ist möglichst simpel gehalten. Er besteht 
    hauptsächlich aus zwei $5\si{\milli\meter}$ dicken Acrylglasplatten, in welcher alle mechanischen 
    Vorrichtungen gelagert sind. Durch diesen Aufbau können Änderungen schnell und 
    einfach angepasst werden. Die Ausrichtung des Abwurfmechanismus erfolgt durch 
    einen flachen Steppermotor, welcher in der drehenden Abwurfeinheit angebracht 
    ist. Dadurch wird die Bauhöhe des Ballwerfers tief gehalten, was einen grossen 
    Vorteil in Sachen Stabilität bietet. Die Drehachse der Abwurfeinheit ist an der Spitze des 
    Ballwerfers mit einem Bolzen angebracht. Somit bleibt die Abwurfposition der 
    Tennisbälle konstant am gleichen Ort. Die Bälle werden durch zwei 
    Beschleunigungsräder beschleunigt, welche jeweils einzeln 
    über eine Übersetzung mit einem Brushlessmotor auf Touren gebracht. Die Beschleunigungsräder 
    drehen gegenläufig, wobei die Tennisbälle zum Abwurf dazwischen hindurchgeführt werden. Die Zuführung zu den Beschleunigungsräder erfolgt mit 
    einem Förderband. Das Förderband transportiert die Bälle mit einer 
    konstanter Geschwindigkeit zu den Beschleunigungsräder, damit alle Tennisbälle die 
    gleiche Startenergie aufweisen. Dadurch ist eine gleichmässige Wurfweite und eine 
    hohe Reproduzierbarkeit gewährleistet.
    Die zwei nachkommenden Bilder zeigen einer Übersicht der verschiedenen weiteren Komponenten, welche im Verlauf der Dokumentation explizit erklärt werden, siehe Abbildungen \ref{abb:Aufsicht auf die linke Seite} und \ref{abb:Frontale Aufsicht}.
	
	\begin{figure}[h!]
	   	\includegraphics[width=1\textwidth,clip,trim=8mm 0mm 15mm 0mm]
	   	{Enddokumentation/Bilder/Geraeteuebersicht_1.jpg}
	   	\centering
	   	\caption{Aufsicht auf die linke Seite}
	   	\label{abb:Aufsicht auf die linke Seite}
	\end{figure}
	
	\begin{figure}[h!]
	   	\includegraphics[width=0.8\textwidth,clip,trim=0mm 0mm 0mm 0mm, angle =-90]
	   	{Enddokumentation/Bilder/Geraeteuebersicht_3.jpg}
	   	\centering
	   	\caption{Frontale Aufsicht}
	   	\label{abb:Frontale Aufsicht}
	\end{figure}
    
    
    \newpage
    \subsection{Herstellung Acrylglas}
		Für die Seitenwände, die Zahnscheibe und für die Verbindungsstege wurde eine 
		$5\si{\milli\meter}$ Dicke Acrylglas (PMMA) Platte verwendet. Die Konturen 
		wurden mit dem Laser an der HSLU gefertigt. Damit der Laser die Daten lesen 
		konnte, wurde aus der CAD-Datei des Seitenprofiles eine dxf-Datei erstellt. Um die Nachbearbeitung 
		möglichst gering zu halten, wurden auch die Lager- und Durchgangsbohrungen 
		für die Schrauben direkt auf das Fertigmass gelasert. Die Lager passten dabei 
		sehr genau in die Bohrungen, sodass sie ein wenig klemmten und trotzdem keine 
		starken Spannungen erzeugten, die allenfalls zu Rissen führen könnten. Dies 
		wurde im Vorfeld getestet (siehe Doku PREN 1). Die einzigen Nachbearbeitungen 
		waren die Bohrungen an den Seitenwänden und an den Spannelementen. 
		Dabei musste auf eine gute Kühlung geachtet werden, da die Restwandstärke nur 
		noch je $1\si{\milli\meter}$ beträgt und das Acrylglas schnell weich wird. 
		Ebenfalls musste die Schnittgeschwindigkeit für das Bohren drastisch gesenkt 
		werden gegenüber einem normalen Kunststoff wie PE. Die Bohrungen an 
		den Seitenwänden wurden auf einer Universalfräsmaschine durchgeführt, die über eine 
		horizontale Frässpindel verfügt (siehe Abbildungen \ref{fig:Acrylglas_1} und 
		\ref{fig:Acrylglas_2}). Die Bohrungen für die 
		Elekronikprints und andere kleine Anpassungen wurden ebenfalls erst nach dem 
		Lasern gefertigt. Zum Beispiel die Ansenkungen für die Schraubenköpfe oder bei 
		den Elekronikprints weil der endgültige Platz noch nicht festgelegt war.
		\begin{figure}[h!]
			\begin{minipage}[hbt]{0.5\textwidth}
		   		\centering
		   		\includegraphics[width=1\textwidth,clip,trim= 0mm 0mm 0mm 0mm]
		   		{Enddokumentation/Bilder/Herstellung_Acrylglas_1.jpg} 
		   		\caption{Herstellung der Acrylglas-Teile}
		   		\label{fig:Acrylglas_1}
			\end{minipage}
			\hfill
			\begin{minipage}[hbt]{0.5\textwidth}
		   		\centering
		   		\includegraphics[width=1\textwidth,clip,trim= 0mm 0mm 0mm 0mm]
		   		{Enddokumentation/Bilder/Herstellung_Acrylglas_2.jpg} 
		   		\caption{Herstellung der Acrylglas-Teile}
		   		\label{fig:Acrylglas_2}
			\end{minipage}
		\end{figure}
		
    \subsection{Hauptcontroller}    	
        Die gesamte Hardware wird über das Freedom-Board\footnote{Development-Board vom 
        Hersteller Freescale, Typ FRDM-KL25Z} gesteuert. Dieses Board hat einen internen UART\footnote{\textbf{U}niversal \textbf{A}synchronous \textbf{R}eceiver 
        \textbf{T}ransmitter, eine serielle Schnittstelle} to USB Converter, über diesen 
        das Smartphone mit dem Freedom-Board kommuniziert. Die Software des Boards besteht aus einem 
        Betriebssystem, dessen Haupttask kontinuierlich den UART Eingangspuffer abruft 
        und die angekommenen Befehle ausführt. Ein wichtiger, nur einmal gebrauchter Task, 
        ist die Initialisierung des Stepperboards, siehe Kapitel \ref{sec:StepperAnsteuerung}. Der Controller 
        verfügt über zwei SPI\footnote{Serial Peripheral Interface, ein synchroner serieller Datenbus }-Schnittstellen. An einer ist das besagte Stepperboard und 
        an der anderen die beiden Brushless-Boards angeschlossen. Weiter wird ein PWM\footnote{pulse width modulation}-Port 
        verwendet, um den Motor der Ballnachführung anzutreiben, siehe Kapitel 
        \ref{sec:Foerderband}.  

   
    \subsection{Spannungsversorgung}       
        \begin{wrapfigure}{r}{0.35\textwidth}
           	\includegraphics[width=0.35\textwidth,clip,trim=0mm 0.5mm 0mm 0mm]
           	{Enddokumentation/Bilder/BeschaltungNetzteile.png}
           	\centering
           	\caption{Schema der Spannungsversorgung} 
           	\label{abb:Spannungsversorgung}
        \end{wrapfigure}
        Gemäss Datenblatt sind die verwendeten Brushless-Motoren mit einer Versorgungsspannung 
        von $12\si{\volt}$ bei einem maximalen Strom von $11\si{\ampere}$ spezifiziert. 
        Der Steppermotor und der DC-Motor sind gemäss Datenblatt für das Betriebsspannung FreeRTOS 
        von $24\si{\volt}$ ausgelegt. Die Spannungsversorgung wird mit zwei Server-Netzteilen 
        realisiert, die je $12\si{\volt}$ mit maximal $60\si{\ampere}$ liefern. Aus 
        Sicherheitsgründen wird für jeden Motor eine separate Sicherung verwendet. Das Schema 
        in Abbildung \ref{abb:Spannungsversorgung} zeigt, wie die Spannungsversorgung 
        umgesetzt ist. Das Freedom-Board wird über USB vom Akkumulator des Smartphone 
        gespeist. Sämtliche Datenblätter zu den verwendeten Motoren sind im Anhangsdokument 
        angefügt. Das zweite Netzteil wurde so umgebaut, dass es potentialfrei arbeitet. So 
        ist es möglich, dass es die $12\si{\volt}$ auf die Spannung des ersten Netzteils 
        generieren kann. Die Sicherung zwischen den beiden Netzteilen ist als Sicherheit 
        hinzugefügt für den Fall, dass es Probleme mit der Potentialfreiheit gibt.	
\newpage        
        \subsection{Smartphone auf dem Ballwerfer}
        Das eingesetzte Smartphone ist ein Samsung Galaxy Nexus I9250. Es führt diverse Aufgaben aus. Dazu zählt 
        die Bildaufnahme, die Auswertung des Bildes, die Berechnung des Winkels, die Kommunikation 
        mit dem Desktop-Computer (Startgerät) und mit dem Freedom-Board.