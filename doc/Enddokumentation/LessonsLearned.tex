\subsection{Lessons Learned}
Eine wichtige Lektion hat das Informatik-Team bezüglich Vorhandensein einer nützlichen, 
konkreten Fallbacklösung (Dank vorhandenem Risikomanagement) für Schlüssel Technologien gelernt. 
Da eine zufriedenstellende Kommunikation mittels Bluetooth nicht gewährleistet werden konnte, 
musste auf die Fallback-Variante, die WLAN als Kommunikationsmittel verwendet, zurückgegriffen werden. 
Das Refactoring des Codes beanspruchte, im Vergleich zur aufgewendet Zeit für das Lösen des Bluetooth-Problems, 
signifikant weniger Zeit. Das Informatik-Team hätte den Umstieg auf die Fallback-Lösung früher einleiten sollen, 
dass hätte einiges an wertvoller Entwicklungszeit eingespart.
\newline
\newline
Als Kommunikationsmittel zwischen den Android-Phone und dem Desktop war anfangs PREN2 eine Verbindung 
mittels Bluetooth gedacht. Obwohl im PREN1 schon einzelne Test bezüglich verbinden der zwei Geräte und 
senden von Daten durchgeführt wurden, kam das Informatik-Team nicht umhin, das Konzept \enquote{Bluetooth} mit einem neuen 
Konzept zu ersetzen. Wie bereits in der Dokumentation des PREN1-Moduls beschrieben, war als Fallback-Lösung 
die Verwendung von WLAN gedacht. 
\\
\\
Die teamübergreifende Zusammenarbeit in der ET-Gruppe hat sich extrem bewährt.  Auf diese Weise 
konnte eine komplexere und anspruchsvolle Lösung realisiert werden. Dementsprechend war der 
Lerneffekt massiv grösser. Die ET-Zusammenarbeit hat sich nicht nur in der Erhöhung der man-power 
niedergeschlagen, sondern auch in der Vielfalt der Themen und deren spezifischen Problemen sowie 
Lösungen. Am Anfang war es zeitintensiv, die Gruppen, die Tools und das gemeinsame Vorgehen zu 
definieren und umzusetzen. Sobald dies erledigt war, funktionierte die Zusammenarbeit innerhalb 
der ET-Gruppe ausserordentlich gut.