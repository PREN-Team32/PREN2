\subsection{Lessons Learned}
Eine wichtige Lektion hat das Informatik-Team bezüglich Vorhandensein einer nützlichen, 
konkreten Fallbacklösung (Dank vorhandenem Risikomanagement) für Schlüssel Technologien gelernt. 
Da eine zufriedenstellende Kommunikation mittels Bluetooth nicht gewährleistet werden konnte, 
musste auf die Fallback-Variante, die WLAN als Kommunikationsmittel verwendet, zurückgegriffen werden. 
Das Refactoring des Codes beanspruchte, im Vergleich zur aufgewendet Zeit für das Lösen des Bluetooth-Problems, 
signifikant weniger Zeit. Das Informatik-Team hätte den Umstieg auf die Fallback-Lösung früher einleiten sollen, 
dass hätte einiges an wertvoller Entwicklungszeit eingespart.
\\
\\
Als Kommunikationsmittel zwischen dem Android-Phone und dem Desktop war anfangs PREN2 eine Verbindung 
mittels Bluetooth gedacht. Obwohl im PREN1 schon einzelne Test bezüglich verbinden der zwei Geräte und 
senden von Daten durchgeführt wurden, kam das Informatik-Team nicht umhin, das Konzept \enquote{Bluetooth} mit einem neuen 
Konzept zu ersetzen. Wie bereits in der Dokumentation des PREN1-Moduls beschrieben, war als Fallback-Lösung 
die Verwendung von WLAN gedacht. 
\\
\\
Die teamübergreifende Zusammenarbeit in der ET-Gruppe hat sich extrem bewährt. Auf diese Weise 
konnte eine komplexere und anspruchsvollere Lösung realisiert werden. Dementsprechend war der 
Lerneffekt massiv grösser. Die ET-Zusammenarbeit hat sich nicht nur in der Erhöhung der \enquote{man-power} 
niedergeschlagen, sondern auch in der Vielfalt der Themen und deren spezifischen Problemen respektive 
Lösungen. Am Anfang war es zeitintensiv, die Gruppen, die Tools und das gemeinsame Vorgehen zu 
definieren und umzusetzen. Sobald dies erledigt war, funktionierte die Zusammenarbeit innerhalb 
der ET-Gruppe ausserordentlich gut.
\\
\\
Von der guten Vorarbeit aus PREN1 konnte enorm profitiert werden. Das komplette CAD-Modell und 
nahezu alle Produktionszeichnungen für die Werkstatt und Laserfertigung waren zu Beginn von PREN2 vorhanden. 
Für Änderungen, Anpassungen oder Weiterentwicklungen musste dennoch viel Zeit investiert werden. 
Dies vor allem, da keines der Teammitglieder grosse Erfahrung in der Handhabung dieser Tools hatte. 
Von diesem Gesichtspunkt her, war es noch wichtiger, dass vieles schon vorhanden war. Auch zeigte sich wie 
wichtig gegenseitige Kontrolle ist. Bestelllisten, Berechnungen etc. von einem Teammitglied erstellt, sollten 
unbedingt gegengelesen werden. Anfänglich gab es Fälle, bei denen dadurch Teile falsch hergestellt wurden. 
Als Konsequenz daraus, wurde alles gegenseitig überprüft, dass solche Fälle nicht mehr vorkamen. Allgemein 
zeigte sich wie wichtig das Überprüfen und Kontrollieren ist. Haben die Teile, die aus der Produktion kommen, 
die gewünschten Masse? Ist die Zeichnung mit den Positionen der Bohrungen noch aktuell oder gab es in der 
Zwischenzeit eine Änderung? Nur auf diese Weise können Fehler vermieden werden und ein planmässiges Gelingen gesichert werden. 
Trotz solcher interner Kontrollen ist man jedoch nicht zu 100\% vor Zwischenfällen in der Materialbeschaffung 
oder Produktion gewappnet. So gab es ein Vorfall, in dem trotz richtiger Bestellung von einem externen Lieferanten 
ein falsches Teil geliefert wurde. Aufgrund vorgegebener Bestellabläufe der Hochschule und Lieferfristen des 
Lieferanten ging es nahezu drei Wochen, bis das neue, richtige Teil geliefert wurde. Dieser Vorfall wiederum 
machte die Wichtigkeit einer guten Projektplanung ersichtlich. So gab es aufgrund des fehlenden Teiles einige 
Verzögerungen im Zusammenbau des Ballwerfers, die jedoch dank der nicht allzu straffen Zeitplanung und 
eingeplanter Puffer nicht so stark ins Gewicht fielen.\\
\\
\\ 
Egal wie detailliert die Planung ist und wie viele Risiken oder allfällige Probleme man bedacht hat, es wird immer unvorhergesehenes passieren. 
In diesen Fällen sind rasches Handeln, gute und manchmal auch kreative Lösungen gefragt. Ein gutes Beispiel hierfür lieferte das Förderband.
Lange war unklar, warum bei den Testversuchen, das mit Bällen bestückte Förderband nicht gleichmässig lief. 
Es brauchte einige Zeit bis klar wurde, dass der Riemen selber das Problem war. (siehe Kapitel \ref{sec:Foerderband})
Da jedoch die Zeit zu weit fortgeschritten war, um ein anderes Fabrikat zu besorgen, musste es mit einfachen Mitteln angepasst werden. 
Kein beteiligtes Teammitglied hatte zuvor bemerkt oder bedacht, dass dies zu einem Problem führen könnte. 
