\subsection{Lessons Learned}
Eine wichtige Lektion hat das Informatik-Team bezüglich Vorhandensein einer nützlichen, 
konkreten Fallbacklösung (Dank vorhandenem Risikomanagement) für Schlüssel Technologien gelernt. 
Da eine zufriedenstellende Kommunikation mittels Bluetooth nicht gewährleistet werden konnte, 
musste auf die Fallback-Variante, die WLAN als Kommunikationsmittel verwendet, zurückgegriffen werden. 
Das Refactoring des Codes beanspruchte, im Vergleich zur aufgewendet Zeit für das Lösen des Bluetooth-Problems, 
signifikant weniger Zeit. Das Informatik-Team hätte den Umstieg auf die Fallback-Lösung früher einleiten sollen, 
dass hätte einiges an wertvoller Entwicklungszeit eingespart.
\newline
\newline
Als Kommunikationsmittel zwischen den Android-Phone und dem Desktop war anfangs PREN2 eine Verbindung 
mittels Bluetooth gedacht. Obwohl im PREN1 schon einzelne Test bezüglich verbinden der zwei Geräte und 
senden von Daten durchgeführt wurden, kam das Informatik-Team nicht umhin, das Konzept \enqute{Bluetooth} mit einem neuen 
Konzept zu ersetzen. Wie bereits in der Dokumentation des PREN1-Moduls beschrieben, war als Fallback-Lösung 
die Verwendung von WLAN gedacht. 