\section{Horizontale Ausrichtung}
    Um die Abwurfeinheit zum Ziel auszurichten, wird ein verstellbarer Mechanismus 
    benötigt, der eine hohe Schrittgenauigkeit aufweist. Der Drehpunkt befindet sich 
    möglichst unter den Schwungrädern, damit die Position des Abwurfes im Zentrum des 
    Spielfeldes bleibt. Die Drehung wird mit einem Schrittmotor realisiert. Die 
    Auflösung des flachen Schrittmotores beträgt 1.8 Grad/Schritt. Ein Schrittmotor 
    ist für diese Anwendung am Besten geeignet, da somit eine sehr exakte Ansteuerung 
    gewährleistet wird. Die Übersetzung des Zahnritzels zur Grundplatte beträgt 
    $1/28$. Somit wird pro Schritt eine Dadurch kann der Verstellwinkel, 
    welcher von der Position des Zieles abhängt, genau eingestellt werden. Der 
    Schrittmotor wird in der Abwurfeinheit angebracht und treibt ein Ritzel an, welches 
    in einen Zahnkranz eingreift. Dadurch kann die Abwurfeinheit gedreht werden. Dies 
    ist in Abbildung \ref{abb:} ersichtlich. Damit die Bauhöhe nicht zusätzlich 
    vergrössert wird, ist der Zahnkranz in die Bodenplatte integriert. Die Bodenplatte 
    mit dem Zahnkranz reicht nicht über die ganze Abwurfeinheit, damit die Masse 
    möglichst klein gehalten werden kann. Der Schrittmotor ist nach dem folgenden 
    Drehmoment von ca. $2Nmm$ ausgelegt. Dies ist sehr klein, da nur der Reibungskoeffizient 
    und die Normalkraft, welche vom Gewicht der Abwurfeinheit abhängt, das Moment erzeugen. 
    Der Reibungskoeffizient wird durch die Lagerung klein gehalten. Die Lagerung erfolgt 
    im Drehzentrum durch eine Hülse und im Endbereich der Abwurfeinheit durch zwei 
    Kugelrollen, welche mit einem seitlichen Abstand angebracht sind. Dadurch wird die 
    Auflagefläche verbreitert und das allfällige Kippen der Abwurfeinheit verhindert. 
    Die Ansteuerung der Schrittmotoren erfolgt über den selbst konstruierten Controller, 
    der als schwarzen Box in Abbildung \ref{abb:} dargestellt ist.
