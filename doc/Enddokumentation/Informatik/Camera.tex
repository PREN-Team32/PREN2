\subsection{Camera}
Als weiteren Baustein der Android-App wird eine Kamera benötigt. Es stehen zurzeit 
zwei Kamera Frameworks von Android zur Verfügung. Das eine Framework ist die (deprecated)‚ 
'Camera', die seit dem Android API-Level\footnote{Application Programming Interface} 1 Teil des 
Android Development Kits ist.
\newline
Das andere Framework ist die Neue, ab API Level 21 verfügbare‚ 
'Camera2'. Um die beiden Camera-Typen zu untersuchen, testen und vergleichen, wurden je eine App programmiert.
Im Vergleich der beiden Frameworks zeigte sich, dass die Entwicklung einer App mit dem alten Camera-Typ 
einiges speditiver von statten geht und der Code viel verständlicher ist. Die Entscheidung fiel deshalb 
auf das zwar veraltete aber erprobte ‚Camera‘-Framework.
\newline
\newline
Erst nach dem Test der beiden Applikationen konnte das Informatik-Team das Gerät für den PREN-Wettkampf festlegen. 
Da dieses Gerät auf Android Jelly Bean (API Level 17) basiert, schliesst dies die Verwendung 
der ‚Camera2‘ (ab API Level 21) aus.
