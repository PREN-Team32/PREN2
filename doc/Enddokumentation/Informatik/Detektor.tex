\subsection{Detektor}
	Für die Bestimmung der Position des Korbes wurde ein Algorithmus 
	eigens entwickelt und in Java implementiert. Auf die Verwendung eines 
	Frameworks (wie beispielsweise OpenCV) wurde verzichtet. Grund dafür 
	liegt in der statischen Problemstellung: Hintergrund, Korbform und -farbe 
	sind immer gleich, was die Problemstellung stark vereinfacht und die Verwendung eines komplexeren Algorithmus hinfällig macht. Somit kann die Erkennungsmechanik einfach und statisch gehalten werden. Unterstützt wird diese Entscheidung auch durch den grösseren Lerneffekt, welche eine Eigenproduktion mit sich bringt. \\
	\\
	Der Algorithmus 
	basiert auf der Tatsache, dass der Korb deutlich dunkler als der 
	Hintergrund ist. Damit mit einem aufgenommenen Bild gearbeitet werden kann, 
	müssen die Ränder abgeschnitten werden. Dies ist notwendig, da die Kamera einen 
	grossen horizontalen Öffnungswinkel aufweist. Dementsprechend geht der 
	Bildbereich links und rechts deutlich über das Spielfeld 
	hinaus, was das Resultat verfälschen könnte. Als zweiter Schritt wird 
	über sämtliche Pixel des Bildes iteriert. Dabei wird für jedes Pixel die 
	Helligkeit anhand einer vordefinierten Schwelle bestimmt, ob es zum 
	Hintergrund (heller) oder zum Korb (dunkel) gehört. Zur Bestimmung der Helligkeit eines Pixels anhand eines RGB Wertes gibt es mehrere Möglichkeiten, welche sich in Genauigkeit und Performance unterscheiden \cite{S:RGB}. Ein genügend hoher 
	Kontrast ist an dieser Stelle entscheidend. Vor allem Schattenwürfe durch 
	seitliche Beleuchtung stellen ein Problem dar. Deshalb wurde bereits in PREN 1 diesbezüglich 
	ein erster Test des entwickelten Prototypen mit zwei Scheinwerfern 
	durchgeführt. Als nächster Schritt 
	wird der Schwerpunkt der dunklen Pixel bestimmt und anhand des gefundenen 
	Schwerpunktes entweder von links oder von rechts her in einem bestimmten 
	horizontalen Bereich (der Korb befindet sich immer auf der selben Höhe)
	über die Pixel iteriert um dabei eine feste Kontur zu finden. Die Fixierung auf einen horizontalen Bereich erhöht zusätzlich die Effizienz des Algorithmus. Diese feste 
	Kontur wird dabei definiert durch eine bestimmte Anzahl weisse Pixel, auf 
	welche wiederum eine Menge schwarzer Pixel folgen muss. Da dieser Prozess 
	immer in derselben horizontalen Ebene stattfindet kann durch eine 
	abschliessende Berechnung der Mittelpunkt des Korbes und der damit 
	verbundene Winkel des Ballwerfers zum Korb trigonometrisch bestimmt werden.
