\subsection{Bluetooth}
Via Bluetooth konnten die serialisierten Objekte zwischen Smartphone-App 
und DesktopViewer ausgetauscht werden. 
Da Java keine native Bluetooth-Kommunikation ermöglicht, wurde auf der 
Desktop Seite auf die BlueCove Library zugegriffen. Da der letzte Release der 
BlueCove Library nicht mehr mit 64-Bit Betriebssystemen funktionierte, wurde ein spezieller Snapshot 
verwendet, welcher auch auf 64-Bit Betriebssystemen lauffähig ist. Allerdings ist dieser Snapshot 
kein offizieller Release.
Android stellt bei der Programmierung eine native Bluetooth 
Library zur Verfügung, weshalb auf der Samrtphone Seite jene direkt verwendet wurde. 
Bei der Implementation kamen einige Probleme auf, was mit der eher dürftigen 
Dokumentation der BlueCove Library, sowie keinerlei vorgängigen Erfahrung in diesem Teilbereich 
erklärt werden konnte. Schlussendlich konnten die serialisierten Objekte, das ConfigItem 
und das ValueItem, zwar gesendet und empfangen werden, allerdings konnte jeweils nur ein Objekt pro aufgebauter 
Bluetooth Verbindung gesendet werden. Falls ein zweites Mal versucht wurde ein ValueItem von der 
Android-App zur Desktop-App zu schicken, wurde das Objekt nicht erkannt, da der 
Stream noch die Informationen des vorher gesendeten ValueItems enthielt. 
Da keine Möglichkeit gefunden wurde, den Stream zu flushen oder eine neue 
Verbindung aufzubauen, ohne das Pairing der Geräte manuell zu bestätigen, wurde Bluetooth 
nicht weiterverfolgt und die Alternative, Websockets (Wireless) gewählt. 

            
