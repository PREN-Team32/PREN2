\subsection{Wireless Kommunikation}

Die Kommunikation findet über Websockets statt und das Verbindungsmedium ist WLAN.
Es wird ein serialisiertes Objekt (ConfigItem) zur Android-App geschickt, welche die erhaltenen 
Daten auswertet und dementsprechend ein Bild aufnimmt. Das Bild wird serialisiert und zur 
Destop-App zurückgeschickt.
Da auf beiden Seiten mit Java gearbeitet wird, kann das Java interne Framework \enquote{Java.net.Socket} verwendet werden. 
Die Android-App fungiert dabei als Server und wartet auf eine eingehende Verbindung, welche von der 
Desktop Seite aus geöffnet wird. Über einen Input- und OutputStream werden die serialisierten Objekte 
über die geöffnete Verbindung gesendet beziehungsweise empfangen. Die Übertragung der Objekte benötigt sehr 
wenig Zeit, selbst wenn grosse Bilder übertragen werden. Die Übertragungsgeschwindigkeit ist abhängig vom verwendetet Netzwerk. 
Da der Desktop den Client stellt, muss dieser die Verbindung via IP auf das Smartphone herstellen, 
dies wurde so gewählt, damit das Smartphone zum Testen und für den realen Durchgang möglichst wenig 
\enquote{in die Hand genommen} werden muss. So muss nur die App gestartet und die Verbindung geöffnet werden, 
danach kann das Smartphone in die Halterung eingefügt werden, ohne das zusätzlich etwas an diesem konfiguriert werden muss. 

            
