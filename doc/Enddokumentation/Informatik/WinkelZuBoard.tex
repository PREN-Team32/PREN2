\subsection{Vom Winkel zum Board}

\begin{figure}[h!]
	\includegraphics[width=0.5\textwidth,clip,trim=20mm 120mm 100mm 40mm]  % trim=l b r t
	{Enddokumentation/Bilder/UebersichtWinkelberechnung.pdf}
	\centering
	\caption{Übersichtsskizze zur Winkelberechnung}
	\label{abb:UebersichtWinkelberechnung}
\end{figure}

Wie in Abbildung \ref{abb:editedPicture} ersichtlich ist, erstellt der Detektor ein Schwarz/Weiss Bild, welches den oberen Rand des Korbs zeigt. 
Um das Gerät auf den Korb auszurichten, muss der Winkel (siehe a in Abbildung \ref{abb:UebersichtWinkelberechnung}) errechnet werden, 
damit man den Winkel anschliessend in die Anzahl MircoSteps des Schrittmotors umrechnen kann.
\newline
\newline
Aus diesem Bild wird nun mittels Trigonometrie der Winkel berechnet. 
Dazu wird zuerst die Länge Gegenkathete (siehe G in Abbildung \ref{abb:UebersichtWinkelberechnung}) 
aus dem Schwarz/Weiss Bild errechnet. Mittels Schwerpunktbestimmung von weissen zu schwarzen Pixel auf dem Bild wird entschieden, 
in welcher Spielhälfte sich der Korb befindet. Ausgehend vom der Mitte des Spielfelds kann die Distanz zum Mittelpunkt des 
Korbs bestimmt werden. Diese Länge liegt zuerst als Anzahl Pixel vor, 
kann anschliessend mit einem (variablen) Umrechnungsverhältnis Pixel zu Zentimeter in Zentimeter umgerechnet werden.
Die Ankathete (siehe A in Abbildung \ref{abb:UebersichtWinkelberechnung}) ist durch das Spielfeld auf die Länge von 170 cm fixiert.
\newline 
Mit diesen beiden Werten lässt sich nun mit dem Arkustangens der resultierende Winkel errechnen.

Dieser Winkel kann mit dem Umrechnungsverhältnis (siehe Kapitel \ref{sec:Aufloesung}) von Microsteps zu Grad 
in die Anzahl Schritte für den Schrittmotor umgewandelt werden. Um die Drehrichtung des Schrittmotors festzulegen, besteht der 
Befehl des Schrittmotors aus einer Drehrichtung und der Anzahl Schritte. Hat der Winkel einen positiven Wert, muss der Anzahl 
Schritte für den Schrittmotor ein  \enquote{r} für die Drehrichtung vorangestellt werden, ist der Winkel negativ, ein  \enquote{f} 
(siehe Abbildung \ref{abb:UebersichtWinkelberechnung}).  
\newline
\newline
\textit{Nähere Angaben zum Schrittmotor und dessen Ansteuerung sind im Anhang unter der Dokumentation des Steppers der ET-Gruppe verfügbar.}
\newline
\newline
Sollte als Beispiel der errechnete Winkel $-7,8\si{\degree}$ sein, würde das Freedom-Board vom Android-Phone folgenden 
Befehl erhalten: f 15320 (Umrechnungsverhältnis: $1\si{\degree}$ entspricht 1964 Microsteps).
\newline
Da die Hypotenuse mit der Länge der Gegenkathete und damit auch mit grösserem Winkel ?? linear, exponentiell,.. ?? an Länge zunimmt, 
hat dies auch eine Verlängerung der Wurfdistanz zur Folge. Um diesem Umstand Rechnung zu tragen, muss die Drehzahl des 
Brushless-DC-Motors entsprechend dem Winkel angepasst werden. Weil das Gerät symmetrisch in der Mitte des Spielfelds platziert ist, 
kann das Vorzeichen des Winkels ausser Acht gelassen werden, denn auf beide Seiten nimmt die Distanz im selben Masse zu.
Dazu wird die folgende Formel verwendet (Die gilt nur bei linearem Anstieg):
\newline
\newline
Formel (muss noch verifiziert werden)
 
\begin{equation}
RPM = RPM_{min} +  \left( \frac{RPM_{max} -RPM_{min}}{Winkel_{max}} \cdot |\text{berechneter Winkel}| \right)
\end{equation}
 
Um den Zeitverlust möglichst klein zu halten, wird die neu errechnete Umdrehungszahl sofort nach erhalten des Winkels an das 
Freedom-Board gesendet, damit die Motoren auf die neue Drehzahl beschleunigt werden können,.
Nachdem das Gerät ausgerichtet ist und die Brushless-DC-Motoren die neu errechnete Drehzahl erreicht haben, kann das Förderband 
für die Ballzuführung eingeschaltet werden. Dies erfolgt mit einem simplen Befehl \enquote{DC setpwm <Wert>}. 
Um alle Motoren, welche noch in Betrieb sind, abzuschalten, muss noch eine Serie von Shutdown-Befehlen an das 
Freedom-Board gesendet werden. 


            
            
