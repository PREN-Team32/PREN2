\section*{Abstract}
In der nachfolgenden Dokumentation wird der Prozess der Konzeptfindung für die Herstellung eines
autonomen Ballwerfers beschrieben. Durch die Aufteilung der Aufgabenstellung in Problembereiche werden
mehrere unterschiedliche Konzepte geschaffen. Von den erstellten Konzepten wurde eines weiter zu einem
Feinkonzept ausgearbeitet, in welchem sämtliche verwendeten Komponenten spezifiziert werden. Als
erstes wird das Startsignal von einem Laptop drahtlos via Bluetooth übertragen. Daraufhin lokalisiert
der fixstehende Ballwerfer den Korb unter Verwendung einer Smartphonekamera, auf welchem eine
entsprechende Applikation zur Korberkennung läuft. Ist die Position einmal bestimmt, wird die Position
an den Controller weitergegeben, welcher den Steppermotor für die Ausrichtung des Werfers betätigt und
anschliessend die Ballzuführung startet. Der Ballwerfer selbst ist statisch und richtet sich an der
Startposition für einen gewinkelten Wurf aus. Einmal ausgerichtet, werden die Bälle einzeln unter
Verwendung von Schwungrädern geworfen.

\newpage