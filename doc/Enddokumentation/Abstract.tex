
\section*{Abstract}
Auf der Grundlage des Konzeptes aus dem Modul PREN 1 ist der autonome Ballwerfer realisiert worden. In der nachfolgenden Dokumentation sind der Prozess der Realisierung, aufgetretene Probleme sowie verschiedene Testberichte beschrieben. Es wird dabei auf das Zusammenspiel von Maschinetechnik, Elektrotechnik und Informatik der einzelnen Komponenten detailliert eingegangen. Ebenfalls erwähnt ist die Herstellung einzelner Schlüsselkomponenten. Die Koordination von Teile bestellen, Werkstücke rechtzeitig in Auftrag geben sowie selber Hand anlegen war einer der grossen Herausforderungen, die es in diesem Modul zu bewältigen gab. Das Ergebnis ist ein autonomer Ballwerfer, der anhand einer Smartphonekamera den Korb erkennt, die Position weitersendet an das Freedomboard, das dann die fixstehende Maschine auf den Korb ausrichtet. Bei laufender Beschleunigungsräder wird die Ballzuführung gestartet und die Bälle in den Korb geworfen. Auch kleine Abweichungen vom Durchmesser der Tennisbälle sind kein Problem für den autonomen Ballwerfer, da dieser sehr schnell angepasst werden kann. Ebenfalls kann die Wurfweite mit verschiedenen Parametern, wie der Drehzahl der Beschleunigungsräder oder der Geschwindigkeit der Ballzuführung, verstellt werden.

