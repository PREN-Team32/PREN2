
\section*{Abstract}
Auf der Grundlage des Konzeptes aus dem Modul PREN 1 wurde ein autonomer Ballwerfer umgesetzt. In der nachfolgenden Dokumentation sind der Prozess der Realisierung, aufgetretene Probleme sowie verschiedene Testfälle beschrieben. Es wird dabei auf das Zusammenspiel von Maschinetechnik, Elektrotechnik und Informatik der einzelnen Komponenten detailliert eingegangen. Ebenfalls erwähnt ist die Herstellung einzelner Schlüsselkomponenten. Die Koordination von Teile bestellen, Werkstücke rechtzeitig in Auftrag geben sowie selber Hand anlegen war einer der grossen Herausforderungen, die es in diesem Modul zu bewältigen gab. Das Ergebnis ist ein autonomer Ballwerfer, der anhand einer Smartphonekamera den Korb erkennt, die Position weitersendet an ein Freedomboard, welches anschliessend die fixstehende Maschine auf den Korb ausrichtet. Bei laufenden Beschleunigungsrädern wird die Ballzuführung gestartet um die Bälle so in den Korb zu werfen. Auch kleine Abweichungen vom Durchmesser der Tennisbälle sind kein Problem für das entstandene Produkt, da dieses unkompliziert entsprechend angepasst werden kann. Ebenfalls kann die Wurfweite mit verschiedenen Parametern, wie der Drehzahl der Beschleunigungsräder oder der Geschwindigkeit der Ballzuführung, verstellt werden.

