\section{Risikomanagement}
Das Risikomanagement hat als Ziel, Probleme bezüglich des Projekts vor dessen Auftreten zu erkennen und Massnahmen
einzuleiten, um deren Risiko zu minimieren oder ganz zu eliminieren.
Das Risikomanagement des PREN Teams 32 umfasst eine Skala, in welcher die Risiken nach deren 
Eintrittswahrscheinlichkeit und Auswirkungsgrad aufgeteilt sind. Die Skalierung beginnt bei kleinem Risiko, markiert mit grüner Farbe 
und steigt an bis zu einem hohen Risiko, markiert mit roter Farbe. Jedes Risiko ist in eine Risikoklasse mit entsprechender Farbe eingeteilt. 
Die Einteilung der Risiken erfolgte durch mehrere Gruppenmitglieder, welche alle ein Schätzung 
bezüglich Eintrittswahrscheinlichkeit und Auswirkungsgrad vorlegten, aus welchen der Mittelwert gebildet wurde.
Das komplette Risikomanagement inklusive der Skalierung ist im Anhangsdokument beigelegt.