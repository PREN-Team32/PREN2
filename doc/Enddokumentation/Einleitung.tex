\section{Einleitung}
Im heutigen Arbeitsumfeld ist es unerlässlich, dass man in der Lage ist in 
einem interdisziplinär zusammengesetzten Team, zu arbeiten. An diesem Punkt 
setzt die Hochschule Luzern Technik \& Architektur mit dem Modul 
\enquote{Produktentwicklung} (PREN) an. Das Ziel dieses Moduls ist, anhand 
einer Aufgabenstellung einen Entwicklungsprozess zu durchlaufen, in einem 
Team eine geeignete Lösung zu eruieren und umzusetzen. Die Teams bestehen 
aus Studierenden aus den Studiengängen Elektrotechnik, Informatik und 
Maschinenbau. Dieses Modul ist in zwei Teile aufgeteilt und erstreckt sich 
über zwei Semester. In PREN 1 wird anhand der Aufgabenstellung ein Konzept 
entwickelt, welches im anschliessenden Semester in PREN 2 umgesetzt wird.\\
\\
In diesem Rahmen erhielten die Teams dieses Jahr die Aufgabe, einen autonomen 
Ballwerfer zu erarbeiten. Das Ziel besteht darin, fünf Tennisbälle, in möglichst 
kurzer Zeit in einen Korb zu befördern. Als weiteres Bewertungskriterium gilt 
das Gewicht des Produkts, welches ab zwei Kilogramm einen stufenweisen 
Punkteabzug zur Folge hat. Das Spielfeld ist sowohl seitlich als auch in der 
Höhe begrenzt. Zusätzlich befindet sich am hinteren Ende eine vertikale Wand, 
vor welcher der Korb auf einer zu dieser Rückwand parallelen Linie platziert 
wird. Die endgültige Position des Korbes wird kurz vor der Abgabe des Startsignals 
durch einen Dozenten festgelegt, und ist somit vor Start nicht bekannt. Die 
Übermittlung des Startsignals muss drahtlos erfolgen, nach Ausführen der Aufgabe, 
muss entweder ein akustisches, oder ein optisches Endsignal ausgegeben werden.\\
\\
Das Ziel der Arbeit ist, diese Aufgabenstellung erfolgreich in ein Produkt 
umzusetzen. Dabei hat sich unser Team eigene Ziele gesetzt und gewichtet. 
Diese sind:
\begin{enumerate}
    \item Treffgenauigkeit
    \item Geschwindigkeit
    \item Gewicht
\end{enumerate}
%Das Produktentwicklungsmodul ist in zwei Teile aufgeteilt, das PREN1-Modul im Herbstsemester sowie das PREN2-Modul im Frühlingssemester. Wichtigste Aufgabe im PREN1-Modul ist das erarbeiten eines Konzepts, eine professionelle, strukturierte Projektabwicklung und das Verifizieren kritischer Teilprobleme mittels Funktionsmuster. Die Realisation des erarbeiteten Konzepts wird im PREN2-Modul in Angriff genommen.

\newpage