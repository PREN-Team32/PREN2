\section{Einleitung}
Im heutigen Arbeitsumfeld ist es unerlässlich, dass man in der Lage ist, in 
einem interdisziplinär zusammengesetzten Team zu arbeiten. An diesem Punkt 
setzt die Hochschule Luzern Technik \& Architektur mit dem Modul 
\enquote{Produktentwicklung} (PREN) an. Das Ziel dieses Moduls ist, anhand 
einer Aufgabenstellung einen Entwicklungsprozess zu durchlaufen, in einem 
Team eine geeignete Lösung zu eruieren und umzusetzen. Die Teams bestehen 
aus Studierenden aus den Studiengängen Elektrotechnik, Informatik und 
Maschinenbau. Dieses Modul ist in zwei Teile aufgeteilt und erstreckt sich 
über zwei Semester. In PREN 1 wird anhand der Aufgabenstellung ein Konzept 
entwickelt, welches im anschliessenden Semester, in PREN 2, umgesetzt wird.\\
\\
In diesem Rahmen erhielten die Teams dieses Jahr die Aufgabe, einen autonomen 
Ballwerfer zu erarbeiten. Das Ziel besteht darin, fünf Tennisbälle in möglichst 
kurzer Zeit in einen Korb zu befördern. Als weiteres Bewertungskriterium gilt 
das Gewicht des Produkts, welches ab zwei Kilogramm einen stufenweisen 
Punkteabzug zur Folge hat. Das Spielfeld ist sowohl seitlich, als auch in der 
Höhe begrenzt. Zusätzlich befindet sich am hinteren Ende eine vertikale Wand, 
vor welcher der Korb auf einer zu dieser Rückwand parallelen Linie platziert 
wird. Die endgültige Position des Korbes wird kurz vor der Abgabe des Startsignals 
durch einen Dozenten festgelegt und ist somit zu Beginn nicht bekannt. Die 
Übermittlung des Startsignals muss drahtlos erfolgen, nach Ausführen der Aufgabe, 
muss entweder ein akustisches oder ein optisches Endsignal ausgegeben werden.\\
\\
In PREN 2 wurde nun das Konzept, welches im ersten Teil von PREN erarbeitet wurde, 
umgesetzt. Das bedeutet, dass ein lauffähiges Funktionsmuster gebaut wurde. 
Ausgehend von den gewonnenen Erkenntnissen des ersten Modulteils, konnten die 
einzelnen Bestandteile gebaut oder umgesetzt werden. In diversen Testläufen 
wurden Systemparameter eruiert und angepasst. Alle Komponenten des 
Gesamtfunktionsmusters wurden aufeinander angepasst und der Ballwerfer konnte in 
Betrieb genommen werden. 