\section{Beschleunigungsräder}
    Die Vortriebskraft für die Tennisbälle wird durch zwei Beschleunigungsräder gewährleistet. Der verwendete
    Beschleunigungsradantrieb wurde aus mehreren Gründen gewählt. Über die Drehzahl und den 
    Anpressdruck der Räder kann die Wurfweite stufenlos eingestellt werden. Somit ist die Maschine 
    für jeden Tennisball in einem bestimmten Bereich gewappnet. Die Beschleunigungsräder wurden 
    aus PVC hergestellt, da dieser Werkstoff einfach zu bearbeiten ist und zugleich eine genügend 
    grosse Festigkeit bietet. Um möglichst viel Gewicht zu sparen, wurde an beiden Planflächen so 
    viel Material wie möglich herausgenommen. Für die Übertragung des Momentes auf die Welle wurde 
    ein Presssitz realisiert. Dies ermöglicht eine gleichmässige Flächenpressung und erhöht die 
    Rundlaufgenauigkeit gegenüber einer geklebten Welle-Nabe-Verbindung. Weiter ist durch die konkave Form der 
    Beschleunigungsräder die Richtung der Ballflugbahn vorgegeben. Hier wird keine zusätzliche Ballführung 
    gebraucht was Gewicht, Kosten und Platz spart. Die Räder sind konkav ausgearbeitet, so dass die 
    Beschleunigung nicht nur über einen Punkt übertragen wird. So wird gewährleistet, dass die Kraft 
    über eine grössere Fläche übertragen wird. Dadurch entsteht wiederum der Vorteil, dass die 
    Beschleunigung geführt abläuft, wodurch ein gerichteter Wurf entsteht. So kann die vorhandene 
    Rotationsenergie vollumfänglich den Tennisbällen übergeben werden. Die Ausrundung wird durch 
    den Radius der Bälle gegeben. Der Durchmesser der Beschleunigungsräder ist so festgelegt, dass 
    mit der vorhandenen Masse ein genügendes Trägheitsmoment zur Verfügung steht. Dies ist nötig, 
    damit bei der Beschleunigung der Tennisbälle die Räder nicht zu stark abgebremst werden. Durch 
    den Durchmesser wird auch die Winkelgeschwindigkeit festgelegt. Zudem sind die Räder mit einer 
    speziellen Haftmatte beschichtet, damit die 
    Kraft optimal auf den Ball übertragen werden kann. Somit wird ein höherer Haftreibungskoeffizient 
    erreicht und Schlupf zwischen Bällen und Beschleunigungsräder verhindert. Die Achsen der zwei Beschleunigungsräder sind im 
    Winkel von 45\si{\degree} zur Bodenplatte angeordnet. Der Abschusswinkel ist so gewählt, dass 
    die Tennisbälle mit einem genügend grossen Einschlagwinkel im Zielbereich landen. So wird die 
    Möglichkeit einer Kollision mit dem Korbrand vermieden.
    \begin{figure}[h!]
       	\includegraphics[width=0.5\textwidth,clip,trim=20mm 5mm 0mm 5mm]
       	{Enddokumentation/Bilder/Abschuss.JPG}
       	\centering
       	\caption{Ballzuführung mit Abschusswinkel}
       	\label{abb:Abschusswinkel}
    \end{figure}
    %    
    \subsection{Antriebsstrang}
        Für die Übertragung der Momente der Brushlessmotoren auf die Beschleunigungsräder wurde eine 
        Zahnradübersetzung von $1:4$ gewählt.
		\begin{figure}[h!]
			\includegraphics[width=0.9\textwidth,clip,trim=0mm 15mm 0mm 0mm]
			{Enddokumentation/Bilder/Antriebsstrang.JPG}
			\centering
			\caption{Der Antriebsstrang}
			\label{abb:Antriebsstrang}
		\end{figure}
        \begin{table}[h!]
            \centering
            \begin{zebratabular}{p{0.10\textwidth}p{0.25\textwidth}}
                \rowcolor{gray} Anzahl & Beschreibung\\
                \rule{0pt}{11pt}2x & Brushlessmotor \\
                \rule{0pt}{11pt}2x & Zahnrad Modul 1.0 Z15\\
                \rule{0pt}{11pt}2x & Zahnrad Modul 1.0 Z30\\
                \rule{0pt}{11pt}2x & Zahnrad Modul 1.25 Z15\\
                \rule{0pt}{11pt}2x & Zahnrad Modul 1.25 Z30\\
            \end{zebratabular}
            \caption{Zahnräder der Übersetzung}
            \label{tab:AntriebsstrangKraft}
        \end{table}
        Auf die Welle des Brushlessmotores wurde eine Büchse eingepresst, da die Standardbohrung des 
        Zahnrades grösser als die des Wellendurchmessers war. Das Zahnrad selber wurde ebenfalls auf die 
        Büchse aufgepresst und noch zusätzlich mit einer Stellschraube fixiert. Das grosse Zahnrad, 
        der Welle der Beschleunigungsräder, wurde ebenfalls nach dem gleichen Schema 
        montiert. Auch hier kam eine Einpressbüchse zum Einsatz und das Zahnrad wurde wiederum mit 
        einer Stellschraube fixiert. Für das Zahnradpaar in der Mitte der Übertragung brauchte es eine 
        zusätzliche Achse. Diese wurde mit zwei Schrauben an der Seitenwand des 
        Acrylglases und an der Motorbefestigungsplatte festgemacht. Die Achse ist fest und dient als 
        Gleitlager. Für die Übertragung der beiden Zahnräder auf dem Gleitlager wurden vier kleine Stangen 
        aus Aluminium verwendet, wobei das Gleitlager selbst ebenfalls aus Aluminium besteht. Die Welle der 
        Beschleunigungsräder ist als Fest- und Loslager ausgeführt. Das Festlager befindet sich auf 
        der Seite des Zahnrades und ist gegen axiales Verschieben mit einem Sicherungsring versehen. 
        Das Übersetzungsverhältnis $i$ berechnet sich wie folgt, der Index 1 bezieht sich dabei auf 
        das Zahnrad am Brushlessmotor.
        \begin{equation}
            i = i_1 \cdot i_2 = \frac{z_2}{z_1} \cdot \frac{z_4}{z_3} = \frac{30}{15} \cdot \frac{30}{15} = 4
        \end{equation}
%
%Ab hier ist es die ET-Doku. Die Files sind Kopien, der Master liegt im ET-Repo
\subsection{Ansteuerung}
Die folgenden Unterkapitel \ref{sec:ET_Hardware} und \ref{sec:ET_Firmware} sind wie im PREN1 \cite{Team32:Doku} in Zusammenarbeit mit der ET-Gruppe erstellt worden. 
\input{Enddokumentation/ET-Gruppe/hardware}
\input{Enddokumentation/ET-Gruppe/firmware}
