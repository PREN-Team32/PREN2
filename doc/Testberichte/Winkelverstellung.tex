\subsection{Windelverstellung}
\begin{zebratabular}{p{4.5cm}p{\textwidth-3.6cm-0.7cm}}
    \rule{0pt}{11pt}\textit{Tester}           & Roger Gisler\\ 
    \rule{0pt}{11pt}\textit{Datum}:           & 17.04.2015\\
    \rule{0pt}{11pt}\textit{Beschreibung}:    & Um den Ballwerfer zum Korb hin auszurichten, ist eine Verstelleinheit montiert. Diese verändert den Winkel des Ballwerfers gegenüber seiner Anfangsposition bis auf einen gewünschten Wert. \\
    \rule{0pt}{11pt}\textit{Akteure}:         & Schrittmotor mit Ritzel welches in den Zahnkranz am Ende der Grundplatte eingreift.\\
    \rule{0pt}{11pt}\textit{Bedingung}:       & Funktionsmuster soweit montierte. Bälle auf Förderband. Mit zusätzlichen Gewichten belastet um Endgewicht von zu simulieren. \\
    \rule{0pt}{11pt}\textit{Erwartete Fehlermeldung}:          & keine \\
    \rule{0pt}{11pt}\textit{Vorgehen}:        & Schrittmotor ansteuern, und in beide Drehrichtungen drehen. Ineinandergreifen der Verzahnung, ruckfreier Lauf, Festigkeit der Bauteile überprüfen. \\
    \rule{0pt}{11pt}\textit{Erwartetes Ergebnis}: & Ruckfreier Lauf, optimaler Eingriff der Verzahnung über den gesamten Verstellwinkel gewährleistet. Verstellung in beide Richtungen. \\
    \rule{0pt}{11pt}\textit{Eingetretenes Ergebnis}: & Ritzel und Zahnrad greifen über den gesamten Verstellwinkel optimal ineinander. Kraft des Schrittmotor ausreichend. 
    Festigkeit des Ritzels aus Acrylglas nicht ausreichend, was zum Bruch des Bauteils führte. 
    Stabilität des gesamten Ballwerfers kritisch da sich ein grosser Teil des Gewichts auf dem vorderen Drehpunkt befindet. \\
    \rule{0pt}{11pt}\textit{Test bestanden?}:     & Nein \\
    \rule{0pt}{11pt}\textit{Weiter Tests nötig?}: & Ja \\
    \rule{0pt}{11pt}\textit{Weiteres Vorgehen}: & Ritzel aus MDF (höhere Elastizität) Lasern.
    Vorderer Quersteg(Acrylglas) auf dem ein grosser Teil des Gewichts lastet aus Aluminium fertigen um die Durchbiegung und Schwingungen zu eliminieren. 
    Grundplatte im vorderen Bereich rund um den Drehpunkt neu Konstruieren um die Kippgefahr zu eliminieren und die Seitliche Stabilität zu erhöhen. \\
\end{zebratabular}  