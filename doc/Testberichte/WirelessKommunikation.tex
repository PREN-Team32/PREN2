\subsection{Wireless-Kommunikation zwischen Smartphone und Desktop-PC}
\begin{zebratabular}{p{4.5cm}p{\textwidth-3.6cm-0.7cm}}
    \rule{0pt}{11pt}\textit{Tester}              & Livio Kunz \\ 
    \rule{0pt}{11pt}\textit{Datum}:           & 30.04.2015   \\
    \rule{0pt}{11pt}\textit{Ort}:             & Teaminsel \\
    \rule{0pt}{11pt}\textit{Beschreibung}:          & Um die Konfigurations- und Resultatsdaten (Items) 
    zwischen dem Smartphone und dem Desktop-PC auszutauschen, verwenden wir Sockets. Mit diesem Test 
    verifiziert man das korrekte bidirektionale übermitteln, entpacken und verpacken der Items.	 \\
    \rule{0pt}{11pt}\textit{Akteure}:          & Operator zur Bedienung des Smartphones \\
    \rule{0pt}{11pt}\textit{Bedingung}:          & Lauffähiges Android-Phone mit 
    Android-Applikation. Desktop-PC mit lauffähiger Desktop-App  \\
    \rule{0pt}{11pt}\textit{Erwartete Fehlermeldung}:          & keine \\
    \rule{0pt}{11pt}\textit{Vorgehen}:          & Applikationen auf beiden Geräten öffnen, Wireless beim 
    Android-Phone einschalten. Verbindung von Desktop Seite starten \\
    \rule{0pt}{11pt}\textit{Erwartetes Ergebnis}:          & Serialisiertes Objekt wird empfangen und gelesen. \\
    \rule{0pt}{11pt}\textit{Eingetretenes Ergebnis}:          & Serialisiertes Objekt wurde gesendet und 
    erfolgreich deserialisiert. Alle Informationen enthalten.\\
    \rule{0pt}{11pt}\textit{Test bestanden?}:          & Ja \\
    \rule{0pt}{11pt}\textit{Weiter Tests nötig?}:          & Nein \\
\end{zebratabular}    
   

   