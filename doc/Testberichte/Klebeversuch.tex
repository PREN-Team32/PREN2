\section{Tests}
\subsection{Klebeversuch}
\begin{tabular}{p{3.6cm}p{\textwidth-3.6cm-0.7cm}}
\rule{0pt}{11pt}\textit{Typ}              & Klebeversuch \\ 
\rule{0pt}{11pt}\textit{Datum}:           & 06.03.2015   \\
\rule{0pt}{11pt}\textit{Ort}:             & Teaminsel \\
\rule{0pt}{11pt}\textit{Tester}:          & Matteo Trachsel \\
\rule{0pt}{11pt}\textit{Ziel des Testes}: & Das Ziel dieses Testes bestand darin, den gekauften Kleber UHU Hart auf seine Klebekraft und auf sein Erscheinungsbild zu testen. \\
\rule{0pt}{11pt}\textit{Aufbau / Ablauf}: & 
\end{tabular}
Für den Test werden verschiedene Acrylglas-Stücke zusammengeklebt.
Hierfür wird der Kleber wie auf der Gebrauchsanweisung auf zwei Verfahren getestet. Im ersten Versuch wird der UHU Kleber aufgetragen und die zwei Platten zusammengeklebt. Im zweiten Versuch wird der Kleber zuerst auf die Acrylglasstücke aufgetragen und gewartet bis er angetrocknet ist, danach noch einmal eine Schicht vom Kleber aufgetragen und zusammengefügt.\\
\rule{0pt}{11pt}\textit{Fazit / Verbesserungs-\newline vorschlag}: & 
Mit dem Versuch konnte gezeigt werden, dass der Kleber sicher glasklar bleibt. Weiter ist die erwünschte Klebekraft bestätigt worden. Beim zweiten Versuch, wo zuerst der Kleber etwas angetrocknet wurde, ist eine deutlich schlechtere Klebekraft festgestellt worden. Dadurch wird der Kleber immer sofort aufgeklebt.\\