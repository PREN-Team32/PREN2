\subsection{Bildaufnahme mit Smartphone inklusive persistente Speicherung}
\begin{zebratabular}{p{4.5cm}p{\textwidth-3.6cm-0.7cm}}
    \rule{0pt}{11pt}\textit{Tester}              & Thomas Wiss \\ 
    \rule{0pt}{11pt}\textit{Datum}:           & 26.03.2015   \\
    \rule{0pt}{11pt}\textit{Ort}:             & Teaminsel \\
    \rule{0pt}{11pt}\textit{Beschreibung}:          & Mittels dem aufgenommenen Bild wird ein 
    Schwarz/Weiss-Abgleich erstellt und der PREN-Korb erkennt. Das Bild wird aufgenommen und 
    anschliessend in einem internen Verzeichnis des Smartphones abgespeichert. 
    Es handelt sich hier um einen Komponententest. \\
    \rule{0pt}{11pt}\textit{Akteure}:          & Operator zur Bedienung des Android-Phones. \\
    \rule{0pt}{11pt}\textit{Bedingung}:          & Lauffähiges Android-Phone mit 
    Android-Applikation \enquote{CameraSD}. \\
    \rule{0pt}{11pt}\textit{Erwartete Fehlermeldung}:          & keine \\
    \rule{0pt}{11pt}\textit{Vorgehen}:          & Applikation öffnen und Button \enquote{Take Picture}
     betätigen. Mittels Android-Explorer den Speicherort des Bildes verifizieren. \\
    \rule{0pt}{11pt}\textit{Erwartetes Ergebnis}:          & Gespeichertes Bild, am 
    vorgängig programmierten Speicherort. \\
    \rule{0pt}{11pt}\textit{Eingetretenes Ergebnis}:          & Bild in guter Qualität am 
    programmierten Speicherort gespeichert. \\
    \rule{0pt}{11pt}\textit{Test bestanden?}:          & Ja \\
    \rule{0pt}{11pt}\textit{Weiter Tests nötig?}:          & Nein \\
\end{zebratabular}    

