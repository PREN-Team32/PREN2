\subsection{Stabilität des mechanischen Aufbaus}
\begin{zebratabular}{p{4.5cm}p{\textwidth-3.6cm-0.7cm}}
    \rule{0pt}{11pt}\textit{Tester}           & Roger Gisler\\ 
    \rule{0pt}{11pt}\textit{Datum}:           & 21.05.2015\\
    \rule{0pt}{11pt}\textit{Beschreibung}:    & Fortsetzung Test \enquote{Winkelverstellung}
    Stabilität des Gesamten Ballwerfers mit den überarbeiteten Teilen überprüfen. \\
    \rule{0pt}{11pt}\textit{Akteure}:         & Komplettes Funktionsmuster \\
    \rule{0pt}{11pt}\textit{Bedingung}:       & Funktionsmuster komplett montiert und mit Bällen bestückt.\\
    \rule{0pt}{11pt}\textit{Erwartete Fehlermeldung}:          & keine \\
    \rule{0pt}{11pt}\textit{Vorgehen}:        & Stabilität, Schwingverhalten, Vibrationen am Ballwerfer unter verschiedensten Lastbedingungen testen.\\
    \rule{0pt}{11pt}\textit{Erwartetes Ergebnis}: & Durchbiegung des vorderen Querstegs eliminiert. Stark verringerte seitliche Schwingungen.\\
    \rule{0pt}{11pt}\textit{Eingetretenes Ergebnis}: & Keine Durchbiegung mehr der vorderen Strebe. Dadurch keine vertikalen Schwingungen mehr. Seitliche Bewegungen oder Schwingungen sind durch die breitere Auflagefläche fast komplett verschwunden.\\
    \rule{0pt}{11pt}\textit{Test bestanden?}:     & Ja\\
    \rule{0pt}{11pt}\textit{Weiteres Tests nötig}: & Nein\\
\end{zebratabular}  