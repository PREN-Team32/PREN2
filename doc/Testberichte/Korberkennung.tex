\subsection{Korberkennung}
\begin{zebratabular}{p{4.5cm}p{\textwidth-5.3cm}}
    \rule{0pt}{11pt}\textit{Tester}              & Niklaus Manser, Livio Kunz, Thomas Wiss \\ 
    \rule{0pt}{11pt}\textit{Datum}:           & 17.05.2015   \\
    \rule{0pt}{11pt}\textit{Ort}:             & Teaminsel \\
    \rule{0pt}{11pt}\textit{Beschreibung}:          & Um die Funktionalität der Korberkennung zu belegen, soll der Detektor in der Lage sein, den Korb in jeder Möglichen Position auf dem Spielfeld zu erkennen.	 \\
    \rule{0pt}{11pt}\textit{Akteure}:          & Operator zur Bedienung der Desktop-App \\
    \rule{0pt}{11pt}\textit{Bedingung}:          & Lauffähiges Android-Phone mit 
    Android-Applikation. Desktop-PC mit lauffähiger Desktop-App  \\
    \rule{0pt}{11pt}\textit{Erwartete Fehlermeldung}:          & keine \\
    \rule{0pt}{11pt}\textit{Vorgehen}:          & Applikationen auf beiden Geräten öffnen, Wireless beim 
    Android-Phone einschalten. Verbindung von Desktop Seite starten. Korb in verschiedenen Positionen aufstellen und den Erkennungsmechanismus starten. \\
    \rule{0pt}{11pt}\textit{Erwartetes Ergebnis}:          & Korb wird erkannt und Position im Bild bestimmt. \\
    \rule{0pt}{11pt}\textit{Eingetretenes Ergebnis}:          & Der Korb konnte in allen gewählten Positionen gefunden werden. \\
    \rule{0pt}{11pt}\textit{Test bestanden?}:          & Ja \\
    \rule{0pt}{11pt}\textit{Weiter Tests nötig?}:          & Ja, Korberkennung soll bis zum Wettbewerb kontinuierlich weiter getestet werden. \\
\end{zebratabular}    
   

   