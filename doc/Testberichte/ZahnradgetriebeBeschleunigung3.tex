\subsection{Zahnradgetriebe der Beschleunigungsräder III}
\begin{zebratabular}{p{4.5cm}p{\textwidth-5.3cm}}
    \rule{0pt}{11pt}\textit{Tester}           & Pascal Roth / Yves Studer\\ 
    \rule{0pt}{11pt}\textit{Datum}:           & 23.04.2015\\
    \rule{0pt}{11pt}\textit{Beschreibung}:    & Wiederholung des Tests Zahnradgetriebe II mit neuer Achse und optimierter Welle-Nabe Verbindung. \\
    \rule{0pt}{11pt}\textit{Akteure}:         & Je zwei Zahnradpaarungen pro Beschleunigungsrad. Brushless-Motoren inkl. deren Ansteuerung. Speziell Achse des Getriebes und Welle-Nabe Verbindung zwischen der Welle der Beschleunigungsräder und dessen Zahnrad.\\
    \rule{0pt}{11pt}\textit{Bedingung}:       & Achsen, Wellen und Zahnräder des Getriebes montiert, Achsabstand auf variabler Seite eingestellt. Motoren montiert, angeschlossen und funktionsfähig.\\
    \rule{0pt}{11pt}\textit{Erwartete Fehlermeldung}:          & keine \\
    \rule{0pt}{11pt}\textit{Vorgehen}:        & Motoren auf Nenndrehzahl beschleunigen. Sitz der erwähnten Welle-Nabe Verbindung unter verschiedenen Betriebsbedingungen und Lastwechsel. Verhalten der Gleitlagerpaarung Alu-Achse Kunststoffzahnrad.\\
    \rule{0pt}{11pt}\textit{Erwartetes Ergebnis}: & Bauteile sind nun den erhöhten Belastungen gewachsen. \\
    \rule{0pt}{11pt}\textit{Eingetretenes Ergebnis}: & Welle-Nabe Verbindung auch nach mehreren Beschleunigungszyklen und Lastwechsel noch fest. Neue Achse aus Aluminium ist der Belastung gewachsen. 
    \\
    \rule{0pt}{11pt}\textit{Test bestanden?}:     & Ja\\
    \rule{0pt}{11pt}\textit{Weiteres Vorgehen}: & Nein\\
\end{zebratabular}  