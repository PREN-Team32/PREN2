\subsection{BLDC-Parameter Test}
\label{sec:ParameterSuche}
\begin{zebratabular}{p{4.5cm}p{\textwidth-5.3cm}}
	\rule{0pt}{11pt}\textit{Tester}           & Yves Studer und Pascal Roth \\ 
	\rule{0pt}{11pt}\textit{Datum}:           & 30.05.2015\\
	\rule{0pt}{11pt}\textit{Beschreibung}:    & Das Ziel des Testes besteht darin, die besten 
	Einstellung für den PID-Regler in den BLDC-Borads, die Drehzahlen der Ballbeschleunigung 
	und die Geschwindigkeit der Förderband Zuführung zu finden. Dies ist ein grosser und 
	aufwändiger Test, in dem jeder Parameter die anderen beeinflusst. Aus diesem Grund ist 
	dieser Test gross.\\
	\rule{0pt}{11pt}\textit{Akteure}:         & Gesamter Aufbau mit Test-Tennisbälle und dem 
	offiziellen Spielfeld\\
	\rule{0pt}{11pt}\textit{Bedingung}:       & Dieser Test musste unter annähernd Wettkampf-
	Bedingungen durchgeführt werden.\\
	\rule{0pt}{11pt}\textit{Erwartete Fehlermeldung}: & keine \\
	\rule{0pt}{11pt}\textit{Vorgehen}:        & Es wurden einzelne Bälle verschossen und die 
	Reaktion des Regler (Überschwingen und Zeit der Ausregelung) beobachtet. Anhand dieser 
	Beobachtungen wurden die entsprechenden Parameter $P$, $I$ und Drehzahl des Reglers 
	angepasst. Ein $D$-Anteil hat zum sofortigen Schwingen des Reglers geführt. Sobald die 
	Grössenordung der Parameter feststanden, kam die Einstellung der Förderbandgeschwindigkeit 
	dazu, was wiederum Einfluss auf die Reglerparameter hatte. Für die Nachführung der Bälle 
	wurden alle 5 Bälle wie unter Wettkampfbedingung sequenziell so schnell wie möglich 
	verschossen. Danach galt es noch den Einfluss der Korbposition (Mitte und am Rand) auf die 
	Drehzahl der Ballbeschleunigung zu ermitteln, damit eine zuverlässige Trefferrate erreicht 
	werden kann.\\
	\rule{0pt}{11pt}\textit{Erwartetes Ergebnis}: & Einen guten Mittelweg zwischen machbarer 
	Regelung und Geschwindigkeit.\\
	\rule{0pt}{11pt}\textit{Eingetretenes Ergebnis}: & Der erste Ball geht immer zu wenig weit. 
	Dieser erste Ball wirkt als Störung auf den Regler worauf dieser hoch regelt. Während dieser 
	Phase werden die anderen Bälle verschossen, die reproduzierbar den Korb treffen. Die erwartete 
	Trefferquote liegt bei 4 von 5 Bälle. Es stellte sich heraus, das ein reiner PI-Regler ohne
	Feedforward am besten ist. die Ballnachführung läuft mit 35\% PWM.\\
	\rule{0pt}{11pt}\textit{Test bestanden?}:     & nahezu ja \\
\end{zebratabular}