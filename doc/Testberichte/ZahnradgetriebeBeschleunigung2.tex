\subsection{Zahnradgetriebe der Beschleunigungsräder II}
\begin{zebratabular}{p{4.5cm}p{\textwidth-5.3cm}}
    \rule{0pt}{11pt}\textit{Tester}           & Pascal Roth / Yves Studer\\ 
    \rule{0pt}{11pt}\textit{Datum}:           & 16.04.2015\\
    \rule{0pt}{11pt}\textit{Beschreibung}:    & Erweiterung des ersten Zahnradgetriebe Tests. Antrieb durch Motoren mit Nenndrehzahl.\\
    \rule{0pt}{11pt}\textit{Akteure}:         & Je zwei Zahnradpaarungen pro Beschleunigungsrad. Brushless-Motoren inkl. deren Ansteuerung.\\
    \rule{0pt}{11pt}\textit{Bedingung}:       & Achsen, Wellen und Zahnräder des Getriebes montiert, Achsabstand auf variabler Seite eingestellt. Motoren montiert, angeschlossen und funktionsfähig.\\
    \rule{0pt}{11pt}\textit{Erwartete Fehlermeldung}:          & keine \\
    \rule{0pt}{11pt}\textit{Vorgehen}:        & Motoren auf Nenndrehzahl beschleunigen. Verhalten des Getriebes bezüglich Schwingungen und Vibrationen überprüfen. Festigkeit der Welle-Nabe Verbindungen.\\
    \rule{0pt}{11pt}\textit{Erwartetes Ergebnis}: & Verhalten des ersten Testlaufs auch mit höherer Drehzahl bestätigt. \\
    \rule{0pt}{11pt}\textit{Eingetretenes Ergebnis}: & Eingriff der Verzahnung auch mit hohen Drehzahlen okay. Keine übermässigen Vibrationen. Welle-Nabe Verbindung Zahnrad – Welle der Beschleunigungsräder ist dem Beschleunigungsmoment nicht gewachsen. Achse aus Gleitlagermaterial iglidur ist der thermischen Belastung durch die Reibung nicht gewachsen.
    \\
    \rule{0pt}{11pt}\textit{Test bestanden?}:     & Verzahnung Ja\newline
    Lagerung Nein\newline
    Welle-Nabe Nein\\
    \rule{0pt}{11pt}\textit{Weiteres Tests nötig}: & Ja\\
    \rule{0pt}{11pt}\textit{Weiteres Vorgehen}: & Defekte Achse aus Aluminium fertigen. Phase in Welle der Beschleunigungsräder Fräsen damit Stellschraube mehr Halt findet. \\
\end{zebratabular}  