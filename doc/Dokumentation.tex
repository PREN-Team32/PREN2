\documentclass[a4paper,10pt,fleqn]{article} % Definiert Papier = A4;
%                                            % Schriftgrösse = 10Punkte;
%                                            % Mathe.-Gl. Modus = linksbündig
%                                            % (siehe http://lefti.amigager.de/latex/Aufbau.html)
%
\usepackage{../common/layout}
\setcounter{tocdepth}{4}   %= Aufnahme in das Inhaltsverzeichnis *
\setcounter{secnumdepth}{4}  % = Nummerierung vertiefen *

\input{Enddokumentation/ET-Gruppe/variables}
\newcommand{\EtPath}{Enddokumentation/ET-Gruppe}

\newcommand{\myTitel}{Realisierung eines\\
autonomen Ballwerfers}
\newcommand{\myDokumentTyp}{Schlussdokumentation}
\setboolean{STANDALONE}{false}
\setboolean{EMBED}{true}
\setlength{\parindent}{0em} 
\begin{document}
    %
    % Deck- und Titelblatt
    %
    \begin{titlepage}
    \begin{center}
        \parindent0pt{\Huge\bfseries \myDokumentTyp}\\[0.5cm]
        {\huge PREN 2}\\
        \begin{figure*}[h!]
            \centering
            \includegraphics[width=0.3\textwidth]{Enddokumentation/Bilder/Logo_lang.JPG}
        \end{figure*}
        Yves Studer\\
        Thomas Wiss\\
        Livio Kunz\\
        Niklaus Manser\\
        Matteo Trachsel\\
        Roger Gisler\\
        Pascal Roth\\
        \vspace*{1cm}
        {\Huge \myTitel}\\[0.5cm]
        \begin{figure*}[h!]
            \includegraphics[width=0.9\textwidth,clip,trim = 30mm 0mm 10mm 0mm]
            {Enddokumentation/Bilder/Titelbild.JPG}
            \centering
        \end{figure*}        
        \vfill{}
        {\normalsize Hochschule Luzern - Technik \& Architektur\\
         PREN 2}\\[0.6cm]
        {\normalsize Horw, Hochschule Luzern - T\&A, \today}
    \end{center}
\end{titlepage}

    \begin{titlepage}
    \begin{center}
        \parindent0pt{\Huge\bfseries \myDokumentTyp}\\[0.5cm]
		{\huge PREN 1, Team 32}\\[2em]
        \begin{tabular}{ll}
            Yves Studer                & Thomas Wiss \\
            Dorfstrasse 28             & Bachhüsliweg 4a \\
            6264 Pfaffnau              & 6042 Dietwil \\
            +41 79 705 48 88           & +41 79 604 93 61 \\
            yves.studer@stud.hslu.ch   & thomas.wiss@stud.hslu.ch \\
                                       & \\
            Livio Kunz                 & Niklaus Manser \\
            Hubelmatt 7                & Brunnmattstrasse 11\\
            6206 Neuenkirch            & 6010 Kriens \\
            +41 79 811 53 03           & +41 77 405 58 56 \\
            livio.kunz@stud.hslu.ch    & niklaus.manser@stud.hslu.ch \\
                                       & \\
            Matteo Trachsel			   & Roger Gisler \\
            Ogimatte 7                 & Eyrüti 16\\
            3713 Reichenbach           & 6467 Schattdorf\\
            +41 79 511 57 88           & +41 79 729 55 34 \\
            matteo.trachsel@stud.hslu.ch & roger.gisler@stud.hslu.ch \\
            						   & \\
            Pascal Roth			       & \\
            Dorfstrasse 18			   & \\
            6275 Ballwil		       & \\
            +41 79 717 68 94	       & \\
            pascal.roth@stud.hslu.ch   & \\
        \end{tabular}\\
        \vspace{3em}
        {\Huge \myTitel}\\[5em]
        Dozent: Markus Thalmann\\[2em]
        Hochschule Luzern - Technik \& Architektur\\   
        Interdisziplinäre Projektarbeit 2014
        \vfill{}
        Horw, Hochschule Luzern - T\&A, \today
    \end{center}
\end{titlepage}
    %
    % Inhaltsverzeichnis umbenennen und anschliessend einen Seitenumbruch
    %  
    
\section*{Abstract}
Auf der Grundlage des Konzeptes aus dem Modul PREN 1 wurde ein autonomer Ballwerfer umgesetzt. In der nachfolgenden Dokumentation sind der Prozess der Realisierung, aufgetretene Probleme sowie verschiedene Testfälle beschrieben. Es wird dabei auf das Zusammenspiel von Maschinetechnik, Elektrotechnik und Informatik der einzelnen Komponenten detailliert eingegangen. Ebenfalls erwähnt ist die Herstellung einzelner Schlüsselkomponenten. Die Koordination von Teile bestellen, Werkstücke rechtzeitig in Auftrag geben sowie selber Hand anlegen war einer der grossen Herausforderungen, die es in diesem Modul zu bewältigen gab. Das Ergebnis ist ein autonomer Ballwerfer, der anhand einer Smartphonekamera den Korb erkennt, die Position weitersendet an ein Freedomboard, welches anschliessend die fixstehende Maschine auf den Korb ausrichtet. Bei laufenden Beschleunigungsrädern wird die Ballzuführung gestartet um die Bälle so in den Korb zu werfen. Auch kleine Abweichungen vom Durchmesser der Tennisbälle sind kein Problem für das entstandene Produkt, da dieses unkompliziert entsprechend angepasst werden kann. Ebenfalls kann die Wurfweite mit verschiedenen Parametern, wie der Drehzahl der Beschleunigungsräder oder der Geschwindigkeit der Ballzuführung, verstellt werden.

 
    \newpage
    \renewcommand{\contentsname}{Inhalt}
    \tableofcontents
    \newpage 
    %
    % Start mit der eigentlicher Arbeit
    %    
    \section{Danksagung}
Das PREN Team 32 wurde während der ganzen Projektphase von verschiedenen Dozenten der Hochschule Luzern Technik \& Architektur unterstützt. 
Ein grosses Dankeschön geht an Herr Markus Thalmann, Herr Ernst Lüthi und Herr Martin Vogel. Sie haben das Team aktiv unterstützt, 
indem sie wertvolle Hinweise und Ratschläge zum Produkt gegeben haben.

    \subsection{Überblick}
Zur Koordination der Abläufe, sowie der Erkennung des Korbes wurden zwei verschiedene Applikationen entwickelt. Einerseits eine Viewer-Applikation, über welche der Ballwerfer konfiguriert und gestartet wird und die erhaltenen Rückgabewerte ausgegeben werden. Des weiteren wurde eine Smartphone-App entwickelt. In dieser Applikation werden die Hauptfunktionalitäten wie beispielsweise die Korberkennung ausgeführt. Der Viewer wird auf einem beliebigen Endgerät wie etwa einem Laptop ausgeführt, wobei die Smartphone-App auf einem Smartphone läuft, welches direkt an der Front des Ballwerfers befestigt wird. Die beiden Applikationen werden via WLAN verbunden und kommunizieren über diese Schnittstelle mittels Data Transfer Objects (DTOs).

\begin{figure}[h!]
	\includegraphics[width=0.8\textwidth,clip,trim=12mm 180mm 15mm 20mm]
	{Enddokumentation/Bilder/Kommunikation_PREN2_v1.pdf}
	\centering
	\caption{Übersicht der beteiligten Geräte}
	\label{abb:UebersichtKommunikation}
\end{figure}

    \newpage
    \section{Von der Idee zum Produkt}
..............
    \section{Grundaufbau}
    Der Ballwerfer ist so konzipiert, das er aus einem fix stehenden Basismodul besteht, 
    welches in der Mitte des Startbereiches positioniert wird. Die Abwurfeinheit, welche 
    den Ballwurfmechanismus und die Ballzuführung beinhaltet, ist auf dem Basismodul 
    drehend gelagert. Weiter ist auch das Smartphone für die Korberkennung und alle 
    Steuereinheiten auf dem Basismodul angebracht. Das Startsignal wird mittels W-Lan 
    von einem externen Labtop übertragen.\\
    Der ganze Aufbau des Ballwurfmechanismus ist sehr simple gehalten. Er besteht 
    hauptsächlich aus zwei 5mm dicken Acrylglasplatten, in welcher alle mechanischen 
    Vorrichtungen gelagert sind. Durch diesen Aufbau können Änderungen schnell und 
    einfach angepasst werden. Die Ausrichtung des Abwurfmechanismus erfolgt durch 
    einen sehr flachen Steppermotor, welcher in der drehenden Abwurfeinheit angebracht 
    ist. Dadurch wird die Bauhöhe des Ballwerfers tief gehalten, was einen grossen 
    Stabilitätsvorteil bietet. Die Drehachse der Abwurfeinheit ist an der Spitze des 
    Ballwerfers mit einem Bolzen angebracht. Somit bleibt die Abwurfposition der 
    Tennisbälle konstant am gleichen Ort. Die Tennisbälle werden durch zwei 
    Beschleunigungsräder beschleunigt. Die Beschleunigungsräder werden je einzeln 
    über eine Übersetzung mit einem Brushlessmotor auf Touren gebracht. Die Schwungräder 
    drehen gegenläufig und die Tennisbälle werden dazwischen hindurchgeführt und 
    abgeworfen. Die Zuführung der Tennisbälle zu den Beschleunigungsräder erfolgt mit 
    einem geregelten Förderband. Das Förderband transportiert die Bälle mit einer 
    konstanter Geschwindigkeit zu den Beschleunigungsräder, damit alle Tennisbälle die 
    gleiche Startenergie aufweisen. Dadurch ist eine gleichmässige Wurfweite und eine 
    hohe Reproduzierbarkeit gewährleistet.\\

    \clearpage
    \section{Softwarearchitektur}  
    \subsection{Überblick}
Zur Koordination der Abläufe, sowie der Erkennung des Korbes wurden zwei verschiedene Applikationen entwickelt. Einerseits eine Viewer-Applikation, über welche der Ballwerfer konfiguriert und gestartet wird und die erhaltenen Rückgabewerte ausgegeben werden. Des weiteren wurde eine Smartphone-App entwickelt. In dieser Applikation werden die Hauptfunktionalitäten wie beispielsweise die Korberkennung ausgeführt. Der Viewer wird auf einem beliebigen Endgerät wie etwa einem Laptop ausgeführt, wobei die Smartphone-App auf einem Smartphone läuft, welches direkt an der Front des Ballwerfers befestigt wird. Die beiden Applikationen werden via WLAN verbunden und kommunizieren über diese Schnittstelle mittels Data Transfer Objects (DTOs).

\begin{figure}[h!]
	\includegraphics[width=0.8\textwidth,clip,trim=12mm 180mm 15mm 20mm]
	{Enddokumentation/Bilder/Kommunikation_PREN2_v1.pdf}
	\centering
	\caption{Übersicht der beteiligten Geräte}
	\label{abb:UebersichtKommunikation}
\end{figure}

    \subsection{Ablauf}

Im Sequenzdiagramm sieht man den Ablauf der ganzen Applikation, sowohl Desktop Seite wie auch Smartphone Seite. Da die Aufgabe vom Gerät völlig autonom verichtet werden muss wird nach dem Start der Android-App das Smartphone nicht mehr bedient werden und alle Interaktionen werden von der Desktop-App gesteuert. Dieser erhält am Ende auch wieder das Schlusssignal.


\begin{figure}[h!]
	\includegraphics[width=0.5\textwidth,clip,trim=20mm 120mm 100mm 40mm]  % trim=l b r t
	{Enddokumentation/Bilder/Sequenzdiagramm_PREN2_v1.pdf}
	\centering
	\caption{Sequenzdiagramm Programmablauf}
	\label{abb:SequenzdiagrammSoftware}
\end{figure}


            

    \subsection{Detektor}
	Für die Bestimmung der Position des Korbes wurde ein Algorithmus 
	eigens entwickelt und in Java implementiert. Auf die Verwendung eines 
	Frameworks (wie beispielsweise OpenCV\footnote{Open Source Computer Vision}) wurde verzichtet. Grund dafür 
	liegt in der statischen Problemstellung: Hintergrund, Korbform und -farbe 
	sind immer gleich, was die Problemstellung stark vereinfacht und die Verwendung eines komplexeren Algorithmus hinfällig macht. Somit kann die Erkennungsmechanik einfach und statisch gehalten werden. Unterstützt wird diese Entscheidung auch durch den grösseren Lerneffekt, welche eine Eigenproduktion mit sich bringt. \\
	\\
	Der Algorithmus 
	basiert auf der Tatsache, dass der Korb deutlich dunkler als der 
	Hintergrund ist. Damit mit einem aufgenommenen Bild gearbeitet werden kann, 
	müssen die Ränder abgeschnitten werden. Dies ist notwendig, da die Kamera einen 
	grossen horizontalen Öffnungswinkel aufweist. Dementsprechend geht der 
	Bildbereich links und rechts deutlich über das Spielfeld 
	hinaus, was das Resultat verfälschen könnte. Als zweiter Schritt wird 
	über sämtliche Pixel des Bildes iteriert. Dabei wird für jedes Pixel die 
	Helligkeit anhand einer vordefinierten Schwelle bestimmt, ob es zum 
	Hintergrund (heller) oder zum Korb (dunkel) gehört. Zur Bestimmung der Helligkeit eines Pixels anhand eines RGB Wertes gibt es mehrere Möglichkeiten, welche sich in Genauigkeit und Performance unterscheiden \cite{S:RGB}. Ein genügend hoher 
	Kontrast ist an dieser Stelle entscheidend. Vor allem Schattenwürfe durch 
	seitliche Beleuchtung stellen ein Problem dar. Deshalb wurde bereits in PREN 1 diesbezüglich 
	ein erster Test des entwickelten Prototypen mit zwei Scheinwerfern 
	durchgeführt. Als nächster Schritt 
	wird der Schwerpunkt der dunklen Pixel bestimmt und anhand des gefundenen 
	Schwerpunktes entweder von links oder von rechts her in einem bestimmten 
	horizontalen Bereich (der Korb befindet sich immer auf der selben Höhe)
	über die Pixel iteriert um dabei eine feste Kontur zu finden. Die Fixierung auf einen horizontalen Bereich erhöht zusätzlich die Effizienz des Algorithmus. 
	\vspace{5mm}
	\begin{figure}[h!]
		\includegraphics[width=0.5\textwidth,clip,trim=10cm 0cm 13cm 0cm]
		{Enddokumentation/Bilder/EditedPicture.jpg}
		\centering
		\caption{zugeschnittenes Schwarz / Weiss Bild des Detektors}
		\label{abb:editedPicture}
	\end{figure}
	\vspace{5mm}
	%
	Diese feste 
	Kontur wird dabei definiert durch eine bestimmte Anzahl weisse Pixel, auf 
	welche wiederum eine Menge schwarzer Pixel folgen muss. Da dieser Prozess 
	immer in derselben horizontalen Ebene stattfindet kann durch eine 
	abschliessende Berechnung der Mittelpunkt des Korbes und der damit 
	verbundene Winkel des Ballwerfers zum Korb trigonometrisch bestimmt werden.

    \subsection{Vom Winkel zum Board}

\begin{figure}[h!]
	\includegraphics[width=0.5\textwidth,clip,trim=20mm 120mm 100mm 40mm]  % trim=l b r t
	{Enddokumentation/Bilder/UebersichtWinkelberechnung.pdf}
	\centering
	\caption{Übersichtsskizze zur Winkelberechnung}
	\label{abb:UebersichtWinkelberechnung}
\end{figure}

Wie in Abbildung \ref{abb:editedPicture} ersichtlich ist, generiert der Detektor ein Schwarz/Weiss Bild, welches den oberen Rand des Korbs zeigt. 
Um das Gerät auf den Korb auszurichten, muss der Winkel (siehe a in Abbildung \ref{abb:UebersichtWinkelberechnung}) berechnet werden, 
damit man den Winkel anschliessend in die Anzahl MircoSteps des Schrittmotors umrechnen kann.
\newline
\newline
Aus diesem Bild wird nun mittels Trigonometrie der Winkel berechnet. 
Dazu wird zuerst die Länge Gegenkathete (siehe G in Abbildung \ref{abb:UebersichtWinkelberechnung}) 
aus dem Schwarz/Weiss Bild berechnet. Ausgehend vom der Mitte des Spielfelds kann die Distanz zum Mittelpunkt des 
Korbs bestimmt werden. Diese Länge liegt zuerst als Anzahl Pixel vor, 
kann anschliessend mit einem (variablen) Umrechnungsverhältnis Pixel zu Zentimeter in Zentimeter umgerechnet werden.
Die Ankathete (siehe A in Abbildung \ref{abb:UebersichtWinkelberechnung}) ist durch das Spielfeld auf die Länge von 170 cm fixiert.
Mit diesen beiden Werten lässt sich nun mit dem Arkustangens der resultierende Winkel errechnen.
Dieser Winkel kann mit dem Umrechnungsverhältnis (siehe Kapitel \ref{sec:Aufloesung}) von Microsteps zu Grad 
in die Anzahl Schritte für den Schrittmotor umgewandelt werden. Um die Drehrichtung des Schrittmotors festzulegen, besteht der 
Befehl des Schrittmotors aus einer Drehrichtung und der Anzahl Schritte. Hat der Winkel einen positiven Wert, muss der Anzahl 
Schritte für den Schrittmotor ein  \enquote{r} für die Drehrichtung vorangestellt werden, ist der Winkel negativ, ein  \enquote{f} 
(siehe Abbildung \ref{abb:UebersichtWinkelberechnung}).  
\newline
\newline
\textit{Nähere Angaben zum Schrittmotor und dessen Ansteuerung sind im Anhang unter der Dokumentation des Steppers der ET-Gruppe verfügbar.}
\newline
\newline
Sollte als Beispiel der errechnete Winkel $-7,8\si{\degree}$ sein, würde das Freedom-Board vom Android-Phone folgenden 
Befehl erhalten: f 15320 (Umrechnungsverhältnis: $1\si{\degree}$ entspricht 1964 Microsteps).
Da die Hypotenuse mit der Länge der Gegenkathete und damit auch mit grösserem Winkel linear an Länge zunimmt, 
hat dies auch eine Verlängerung der Wurfdistanz zur Folge. Um diesem Umstand Rechnung zu tragen, muss die Drehzahl des 
Brushless-DC-Motors entsprechend dem Winkel angepasst werden. Weil das Gerät symmetrisch in der Mitte des Spielfelds platziert ist, 
kann das Vorzeichen des Winkels ausser Acht gelassen werden, denn auf beide Seiten nimmt die Distanz im selben Masse zu.
Dazu wird die folgende Formel verwendet (Die gilt nur bei linearem Anstieg):
\newline
\newline
Formel (muss noch verifiziert werden)
 
\begin{equation}
RPM = RPM_{min} +  \left( \frac{RPM_{max} -RPM_{min}}{Winkel_{max}} \cdot |\text{berechneter Winkel}| \right)
\end{equation}
 
Um den Zeitverlust möglichst klein zu halten, wird die neu errechnete Umdrehungszahl sofort nach erhalten des Winkels an das 
Freedom-Board gesendet, damit die Motoren auf die neue Drehzahl beschleunigt werden können.
Nachdem das Gerät ausgerichtet ist und die Brushless-DC-Motoren die neu errechnete Drehzahl erreicht haben, kann das Förderband 
für die Ballzuführung eingeschaltet werden. Dies erfolgt mit einem simplen Befehl \enquote{DC setpwm <Wert>}. 
Um die beiden Brushless-Motoren und den DC-Motor des Förderbands abzuschalten, muss noch eine Serie von Shutdown-Befehlen an das 
Freedom-Board gesendet werden. 


            
            

	\subsection{Bluetooth}
Via Bluetooth konnten die serialisierten Objekte zwischen Android-App 
und Desktop-App gesendet und empfangen werden. 
Da Java keine native Bluetooth Kommunikation ermöglicht, wurde auf der 
Desktop Seite auf die BlueCove Library zugegriffen. Da der letzte Release der 
BlueCove Library nicht mehr mit 64-Bit Betriebssystemen funktionierte, wurde ein Snapshot 
verwendet, welcher auch auf 64-Bit Betriebssystemen funktioniert. Allerdings ist dieser Snapshot 
kein \enquote{offizieller} Release. \newline
Google stellt bei der Programmierung eine native Bluetooth 
Library zur Verfügung, weshalb auf der Samrtphone Seite jene direkt verwendet wurde. 
Bei der Implementation kamen einige Probleme auf, was mit der eher dürftigen 
Dokumentation der BlueCove Library, sowie keinerlei vorgängigen Erfahrung in diesem Teilbereich 
erklärt werden konnte. Schlussendlich konnten die serialisierten Objekte, das ConfigItem 
und das ValueItem, gesendet beziehungsweise empfangen werden.
Allerdings konnte jeweils nur ein Objekt pro aufgebauter 
Bluetooth Verbindung gesendet werden. Falls ein zweites Mal versucht wurde ein ValueItem von der 
Android-App zur Desktop-App zu schicken, wurde das Objekt nicht erkannt, da der 
Stream noch die Informationen des vorher gesendeten ValueItems enthielt. 
Da keine Möglichkeit gefunden wurde, den Stream zu flushen oder eine neue 
Verbindung aufzubauen, ohne das Pairing der Geräte manuell zu bestätigen, wurde Bluetooth 
nicht weiterverfolgt und die Alternative, Websockets (Wireless) gewählt. 

            
    
    \subsection{WLAN Kommunikation}

Die Kommunikation findet über Websockets statt und das Verbindungsmedium ist WLAN.
Es wird ein serialisiertes Objekt (ConfigItem) zur Android-App geschickt, welche die erhaltenen 
Daten auswertet und dementsprechend ein Bild aufnimmt. Das Bild wird serialisiert und zur 
Destop-App zurückgeschickt.
Da auf beiden Seiten mit Java gearbeitet wird, kann das Java interne Framework \enquote{Java.net.Socket} verwendet werden. 
Die Android-App fungiert dabei als Server und wartet auf eine eingehende Verbindung, welche von der 
Desktop Seite aus geöffnet wird. Über einen Input- und OutputStream werden die serialisierten Objekte 
über die geöffnete Verbindung gesendet beziehungsweise empfangen. Die Übertragung der Objekte benötigt sehr 
wenig Zeit, selbst wenn grosse Bilder übertragen werden. Die Übertragungsgeschwindigkeit ist abhängig vom verwendetet Netzwerk. 
Da der Desktop den Client stellt, muss dieser die Verbindung via IP auf das Smartphone herstellen, 
dies wurde so gewählt, damit das Smartphone zum Testen und für den realen Durchgang möglichst wenig 
\enquote{in die Hand genommen} werden muss. So muss nur die App gestartet und die Verbindung geöffnet werden, 
danach kann das Smartphone in die Halterung eingefügt werden, ohne das zusätzlich etwas an diesem konfiguriert werden muss. 

            

    \subsection{Model}
Das Model beinhaltet die Klassen welche serialisiert werden und vom Desktop zum Smartphone und zurück gesendet werden. \ref{abb:Kontextdiagramm}
Das Model wird als Zwischenspeicher für die benötigten Informationen, welche ausgetauscht werden müssen, benötigt.
\subsubsection{ValueItem}
Das ValueItem ist eine serialisierbare Singleton-Klasse, welche von der Android-App 
zur Desktop-App gesendet wird. Die Klasse enthält die Informationen, ob der Korb 
erkennt wurde, wie lange die Erkennung benötigte und das originale, wie auch das editierte Bild. 
Diese Informationen werden von der Desktop-App ausgelesen und dargestellt. 

\subsubsection{ConfigItem}
Das ConfigItem ist eine serialiserbare Singleton-Klasse, welche zur Kommunikation 
zwischen Desktop-App und Android-App genutzt wird. Die Klasse dient als Datenkonstrukt 
um Konfigurationswerte, die für die Korberkennung benötigt werden, an die Android-App zu senden.

            
            
    
    \subsection{USB Connection}
Um verschiedene Befehle und den vom Detektor berechneten Winkel vom Android-Phone an das Freedom-Board (KL25Z) 
zu senden, wird eine USB\footnote{Universal Serial Bus}-Schnittstelle zwischen diesen zwei Geräte verwendet.
\newline
Da Android keine native Anbindung an MicroController-Boards via USB zur Verfügung stellt, muss auf 
Frameworks von Drittanbietern zugegriffen werden. In unserem Fall verwenden wir UsbSerial \cite{Inf:UsbSerial} als support-library, wie im Namen ersichtlich, 
handelt es sich dabei um eine virtuelle serielle Schnittstelle (COM-Schnittstelle \footnote{Component Object Model})
für den USB-Port eines Android-Gerätes. Dieser Serial Controller erlaubt es programmatisch alle nötigen Konfigurationen 
(Baudrate, Databit, Stopbits, Paritybit, Flowcontrol) für das Board zu setzen. Mittels einer einzigen 
Methode können nun anschliessend Bytes gesendet werden. Um ein ressourcenfressendes Pollen am Eingang zu vermeiden, 
bietet die Library eine Callback-Funktion, die aufgerufen wird, wenn Daten am Eingang eintreffen.
\newline
Mit dem Programmieren und Erstellen eines ersten Prototyps (Android App, siehe Abbildung \ref{abb:ScreenshotSerialPortExample}) musste die Frage geklärt werden, 
ob mit der support-library das Senden eines einzelnen Strings (Char) zum Freedom-Board und das Echo des 
Boards auf dem Bildschirm des Smartphones ausgegeben werden kann. Nach einem Testlauf zeigte sich, dass
das Senden des Chars soweit nachvollziehbar einwandfrei funktioniert. Beim Empfangen der Daten zeigte 
sich, dass beim parsen der Bytes im Code der support-library ein oder zwei Bits falsch gesetzt wurden. 
Das hatte zur Folge, dass gewisse Buchstaben oder Zahlen nicht mehr korrekt in ASCII 
\footnote{American Standard Code for Information Interchange} respektive UTF-8 
\footnote{Universal Coded Character Set + Transformation Format—8-bit}
umgewandelt werden konnten. Bei einer Handvoll Buchstaben und Zahlen (f, z, o, w, 2-8) funktionierte die
Echo-Methode einwandfrei, bei allen anderen Zeichen jedoch nicht.
\newline
\begin{figure}[h!]
	\includegraphics[width=0.3\textwidth,clip,trim=0mm 0mm 0mm 0mm]
	{Enddokumentation/Bilder/Screenshot_SerialPortExample_debug.png}
	\centering
	\caption{Screenshot Prototypen-App zur USB Kommunikation}
	\label{abb:ScreenshotSerialPortExample}
\end{figure}
\newline
Nach einem Update Mitte Februar 2015 durch den Entwickler der support-library funktioniert die App nun 
in bidirektionaler Richtung ohne Einschränkungen. Teil des Updates (respektive einziger Grund) war die 
Erweiterung der unterstützten Geräte um die CH34xSerialDevice-Schnittstelle. Damit erklärt sich auch das 
Fehlverhalten der  App vor dem Update, als die Bits nur bruchstückweise richtig geparst wurden. Es lag 
an der damals schon vorhandenen Schnittstelle zu der CDC-Geräte-Gruppe \footnote{Communication Device Class}, 
die nicht alle Geräte voll, sondern nur teilweise unterstütze und zu einem dieser nicht vollumfänglich 
unterstützen Typen gehörte das Freedom-Board. \newline
Der Code aus dem  Prototyp-App wird nun gewissermassen als Komponente zur Kommunikation mit dem USB-Gerät
im Android-App verwendet. 
\newline
\newline
Die Logik und Auslösung der Befehle an die Motoren findet vom Android-App via Freedom-Board statt. Damit 
sind das Starten der Brushless-DC-Motoren, das Ausrichten des Geräts mit dem Schrittmotor und das Starten 
des Motors für das Förderband gemeint. Nach dem Ausführen ihrer Aufgabe müssen die betreffeden Motoren 
ausgeschaltet werden, das wiederum vom Android-Phone aus passiert.
\newline
\newline
// evtl Code-Snippet??

    \subsection{Camera}
Als weiteren Baustein der Android-App wird eine Kamera benötigt. Es stehen zurzeit 
zwei Kamera Frameworks von Android zur Verfügung. Das eine Framework ist die (deprecated)‚ 
'Camera', die seit dem Android API-Level\footnote{Application Programming Interface} 1 Teil des 
Android Development Kits ist.
\newline
\begin{figure}[h!]
	\includegraphics[width=0.3\textwidth,clip,trim=0mm 0mm 0mm 0mm]
	{Enddokumentation/Bilder/Screenshot_CameraSD.png}
	\centering
	\caption{Screenshot Camera Prototypen-App}
	\label{abb:ScreenshotCameraSD}
\end{figure}
Das andere Framework ist die Neue, ab API Level 21 verfügbare‚ 
 \enquote{Camera2}. Um die beiden Camera-Typen zu untersuchen, testen und vergleichen, wurden je eine App programmiert.
Im Vergleich der beiden Frameworks zeigte sich, dass die Entwicklung einer App mit dem alten Camera-Typ 
einiges speditiver von statten geht und der Code viel verständlicher ist. Die Entscheidung fiel deshalb 
auf das zwar veraltete aber erprobte  \enquote{Camera}-Framework.
\newline
\newline
Erst nach dem Test der beiden Applikationen konnte das Informatik-Team das Gerät für den PREN-Wettkampf festlegen. 
Da dieses Gerät auf Android Jelly Bean (API Level 17) basiert, schliesst dies die Verwendung 
der  \enquote{Camera2} (ab API Level 21) aus.
  
    \section{Horizontale Ausrichtung}
    Die Ansteuerung des Stepper-Motors erfolgt über die entwickelte Hardware der PREN-ET-Gruppe. Dieses Board kann direkt auf das Freedom-Board aufgesteckt und über die Konsole bedient werden. Die gesamte Dokumentation dazu ist im Anhang \ref{apx:} angefügt.
    
    \section{Förderband}
\label{sec:Foerderband}
	Da die Beschleunigungsräder durch den Abwurf abgebremst werden, müssen 
	sie nach jedem Wurf wieder auf Nenndrehzahl gebracht werden. Um dafür 
	genügend Zeit zu haben, erfolgt die Zuführung der Bälle in zeitlichen 
	Abständen. Ein weiteres Kriterium für eine konstante Wurfweite, ist eine 
	gleichbleibende Geschwindigkeit mit der die Bälle zwischen die 
	Beschleunigungsräder kommen. Der Antrieb des Förderbandes erfolgt mit 
	einem DC-Motor, siehe Kapitel \ref{sec:FoerderbandAnsteuerung}. Die 
	Drehzahl ist mittels einer Zahnradpaarung mit $i=5$ übersetzt, um das 
	benötigte Drehmoment an die Antriebswelle des Förderbandes zu übertragen. 
	Die Welle und die Achse sind einteilig aus Aluminium gedreht und mittels Kugellager in den Seitenplatten gelagert. 
	Die Auflagefläche des Riemens auf der Antriebswelle ist bombiert gefertigt. 
	Dadurch wird ein seitliches Abrutschen des Riemens im Betrieb verhindert. 
	Auf dem Förderband, welches ein Flachbandriemen ist, sind Führungsschaufeln 
	angebracht, siehe Abbildung \ref{abb:Foerderband}. Durch den Abstand dieser 
	Schaufeln ergeben sich die zum Hochdrehen der Motoren verfügbaren Zeitintervalle. Die Führungsschaufeln sind so 
	ausgerundet, dass der Ball möglichst lange geführt werden kann ohne die 
	Beschleunigungsräder zu berühren. Sie sind aus $1\si{\milli\meter}$ 
	Aluminium Blech gefertigt und wurden auf dem Riemen aufgeklebt. Bei den 
	Testversuchen stellte sich heraus, dass durch die aufgeklebten Schaufeln 
	der Riemen nicht rund läuft. Jedes Mal wenn eine Klebestelle die Achse 
	passierte, erhöhte sich der Widerstand und das Band drehte langsamer. 
	Ausserdem hat der Endlosriemen an der Stelle an der er gefügt ist eine 
	höhere Steifigkeit, was denselben Effekt hatte. Ausgehend von diesen 
	Erkenntnissen wurde der Riemen neu gefertigt. Die Schaufeln wurden nicht 
	mehr geklebt, sondern mit einem Faden angenäht. Dazu wurden in Riemen 
	und Führungsschaufeln je drei Bohrungen gemacht und mit verstärktem Faden 
	Verbunden. So sind die Führungsschaufeln nur noch mit einer Linienverbindung 
	und nicht mehr mit einer Flächenverbindung auf dem Riemen befestigt. An 
	der Fügestelle des Riemens wurde in Längsrichtung Material entnommen, was 
	eine Verringerung der Steifigkeit bewirken sollte. In den nachfolgenden 
	Testversuchen zeigten die Anpassungen ihre gewünschte Wirkung. Die 
	Geschwindigkeit während einer Umdrehung war nun annähernd konstant. Aus 
	diversen Testversuchen der Ballzuführung während PREN1 wurde erkannt, dass 
	für einen idealen Abwurf die Tennisbälle mit beiden Beschleunigungsräder 
	gleichzeitig in Kontakt kommen müssen. Somit ist es notwendig die Bälle 
	zunächst unter dem oberen Beschleunigungsrad hindurch und anschliessend in 
	einem 45\si{\degree} Winkel nach oben zuzuführen. Dazu dient ein Führungselement, 
	siehe Abbildung \ref{abb:Abschusswinkel}
	welches auf beiden Seiten des Acrylglases angebracht ist. Diese 
	Führungselemente sind an die Form der Tennisbälle angepasst und mittels 
	3D Druck hergestellt worden. 
	\begin{figure}[h!]
    	\includegraphics[width=0.9\textwidth,clip,trim=0mm 0mm 0mm 0mm]
    	{Enddokumentation/Bilder/Foerderband.jpg}
    	\centering
    	\caption{Aufbau des Förderbandes}
    	\label{abb:Foerderband}
 	\end{figure}
\newpage
\subsection{Ansteuerung DC-Motor}
\label{sec:FoerderbandAnsteuerung}
    Die Ansteuerung des Motors, der das Förderband antreibt, erfolgt mittels PWM. Auf diese 
    Weise lässt sich die Drehzahl und somit die Nachführgeschwindigkeit einstellen. Das 
    Band muss nur in eine Richtung angetrieben werden, wodurch die Ansteuerung einfacher 
    realisiert werden kann. Das Schema ist in Abbildung \ref{abb:SchemaAnsteuerung} 
    ersichtlich. Im Wesentlichen besteht diese Ansteuerung aus einem Vortreiber und einem 
    Schalter. Der Treiber bewirkt ein möglichst schnelles und effizientes Öffnen und Schliessen des Schalters. Auf diese Weise reduziert man die Schaltverluste. 
    \begin{figure}[h!]
    	\includegraphics[width=0.7\textwidth,clip,trim=0mm 2mm 0mm 7mm]
    	{Enddokumentation/Bilder/Schema_DC-Ansteuerung.png}
    	\centering
    	\caption{Schema des Förderbandansteuerung}
    	\label{abb:SchemaAnsteuerung}
    \end{figure}
    \section{Beschleunigungsräder}
    Die Vortriebskraft für die Tennisbälle wird durch zwei Beschleunigungsräder gewährleistet. Der verwendete
    Beschleunigungsradantrieb wurde aus mehreren Gründen gewählt. Über die Drehzahl und den 
    Anpressdruck der Räder kann die Wurfweite stufenlos eingestellt werden. Somit ist die Maschine 
    für jeden Tennisball in einem bestimmten Bereich gewappnet. Die Beschleunigungsräder wurden 
    aus PVC hergestellt, da dieser Werkstoff einfach zu bearbeiten ist und zugleich eine genügend 
    grosse Festigkeit bietet. Um möglichst viel Gewicht zu sparen, wurde an beiden Planflächen so 
    viel Material wie möglich herausgenommen. Für die Übertragung des Momentes auf die Welle wurde 
    ein Presssitz realisiert. Dies ermöglicht eine gleichmässige Flächenpressung und erhöht die 
    Rundlaufgenauigkeit gegenüber einer geklebten Welle-Nabe-Verbindung. Weiter ist durch die konkave Form der 
    Beschleunigungsräder die Richtung der Ballflugbahn vorgegeben. Hier wird keine zusätzliche Ballführung 
    gebraucht was Gewicht, Kosten und Platz spart. Die Räder sind konkav ausgearbeitet, so dass die 
    Beschleunigung nicht nur über einen Punkt übertragen wird. So wird gewährleistet, dass die Kraft 
    über eine grössere Fläche übertragen wird. Dadurch entsteht wiederum der Vorteil, dass die 
    Beschleunigung geführt abläuft, wodurch ein gerichteter Wurf entsteht. So kann die vorhandene 
    Rotationsenergie vollumfänglich den Tennisbällen übergeben werden. Die Ausrundung wird durch 
    den Radius der Bälle gegeben. Der Durchmesser der Beschleunigungsräder ist so festgelegt, dass 
    mit der vorhandenen Masse ein genügendes Trägheitsmoment zur Verfügung steht. Dies ist nötig, 
    damit bei der Beschleunigung der Tennisbälle die Räder nicht zu stark abgebremst werden. Durch 
    den Durchmesser wird auch die Winkelgeschwindigkeit festgelegt. Zudem sind die Räder mit einer 
    speziellen Haftmatte beschichtet, damit die 
    Kraft optimal auf den Ball übertragen werden kann. Somit wird ein höherer Haftreibungskoeffizient 
    erreicht und Schlupf zwischen Bällen und Beschleunigungsräder verhindert. Die Achsen der zwei Beschleunigungsräder sind im 
    Winkel von 45\si{\degree} zur Bodenplatte angeordnet. Der Abschusswinkel ist so gewählt, dass 
    die Tennisbälle mit einem genügend grossen Einschlagwinkel im Zielbereich landen. So wird die 
    Möglichkeit einer Kollision mit dem Korbrand vermieden.
    \begin{figure}[h!]
       	\includegraphics[width=0.5\textwidth,clip,trim=20mm 5mm 0mm 5mm]
       	{Enddokumentation/Bilder/Abschuss.JPG}
       	\centering
       	\caption{Ballzuführung mit Abschusswinkel}
       	\label{abb:Abschusswinkel}
    \end{figure}
    %    
    \subsection{Antriebsstrang}
        Für die Übertragung der Momente der Brushlessmotoren auf die Beschleunigungsräder wurde eine 
        Zahnradübersetzung von $1:4$ gewählt.
		\begin{figure}[h!]
			\includegraphics[width=0.9\textwidth,clip,trim=0mm 15mm 0mm 0mm]
			{Enddokumentation/Bilder/Antriebsstrang.JPG}
			\centering
			\caption{Der Antriebsstrang}
			\label{abb:Antriebsstrang}
		\end{figure}
        \begin{table}[h!]
            \centering
            \begin{zebratabular}{p{0.10\textwidth}p{0.25\textwidth}}
                \rowcolor{gray} Anzahl & Beschreibung\\
                \rule{0pt}{11pt}2x & Brushlessmotor \\
                \rule{0pt}{11pt}2x & Zahnrad Modul 1.0 Z15\\
                \rule{0pt}{11pt}2x & Zahnrad Modul 1.0 Z30\\
                \rule{0pt}{11pt}2x & Zahnrad Modul 1.25 Z15\\
                \rule{0pt}{11pt}2x & Zahnrad Modul 1.25 Z30\\
            \end{zebratabular}
            \caption{Zahnräder der Übersetzung}
            \label{tab:AntriebsstrangKraft}
        \end{table}
        Auf die Welle des Brushlessmotores wurde eine Büchse eingepresst, da die Standardbohrung des 
        Zahnrades grösser als die des Wellendurchmessers war. Das Zahnrad selber wurde ebenfalls auf die 
        Büchse aufgepresst und noch zusätzlich mit einer Stellschraube fixiert. Das grosse Zahnrad, 
        der Welle der Beschleunigungsräder, wurde ebenfalls nach dem gleichen Schema 
        montiert. Auch hier kam eine Einpressbüchse zum Einsatz und das Zahnrad wurde wiederum mit 
        einer Stellschraube fixiert. Für das Zahnradpaar in der Mitte der Übertragung brauchte es eine 
        zusätzliche Achse. Diese wurde mit zwei Schrauben an der Seitenwand des 
        Acrylglases und an der Motorbefestigungsplatte festgemacht. Die Achse ist fest und dient als 
        Gleitlager. Für die Übertragung der beiden Zahnräder auf dem Gleitlager wurden vier kleine Stangen 
        aus Aluminium verwendet, wobei das Gleitlager selbst ebenfalls aus Aluminium besteht. Die Welle der 
        Beschleunigungsräder ist als Fest- und Loslager ausgeführt. Das Festlager befindet sich auf 
        der Seite des Zahnrades und ist gegen axiales Verschieben mit einem Sicherungsring versehen. 
        Das Übersetzungsverhältnis $i$ berechnet sich wie folgt, der Index 1 bezieht sich dabei auf 
        das Zahnrad am Brushlessmotor.
        \begin{equation}
            i = i_1 \cdot i_2 = \frac{z_2}{z_1} \cdot \frac{z_4}{z_3} = \frac{30}{15} \cdot \frac{30}{15} = 4
        \end{equation}
%
%Ab hier ist es die ET-Doku. Die Files sind Kopien, der Master liegt im ET-Repo
\subsection{Ansteuerung}
Die folgenden Unterkapitel \ref{sec:ET_Hardware} und \ref{sec:ET_Firmware} sind wie im PREN1 \cite{Team32:Doku} in Zusammenarbeit mit der ET-Gruppe erstellt worden. 
\input{Enddokumentation/ET-Gruppe/hardware}
\input{Enddokumentation/ET-Gruppe/firmware}

    \section{Projektplanung / -Management}

Das Projektteam 32 besteht aus sieben Personen die sich auf folgende Studienrichtungen aufteilen: Drei Personen Maschinentechnik, drei Personen Informatik und eine Person Elektrotechnik. Die Studienrichtungen sind sogleich die jeweiligen Verantwortungen. In den Bereichen mit mehreren Projektmitgliedern wird die Verantwortung für Teilaufgaben jeweils situativ verteilt. Für allgemeine Projektarbeiten ist jeweils die Hauptverantwortliche Person bestimmt. Diese kann Teilaufgaben definieren und sie an andere Teammitglieder zur Bearbeitung delegieren. Die Hierarchie im Team ist bewusst flach und ohne eigentlichen Projektleiter gehalten. Entscheide werden im Plenum diskutiert und gefällt. Die Leitung oder Führung einer Besprechung obliegt der oder den Verantwortlichen des jeweiligen Themas. Mit dieser Teamstruktur ist gewährleistet, dass alle Mitglieder Verantwortung tragen können und müssen. Dies soll Motivation und Eigeninitiative fördern. Im Verlauf von PREN1 hat sich gezeigt, dass diese Projektorganisation optimal ist. Jedes Teammitglied hatte das gleiche Mitspracherecht, was die Kreativität und das Engagement wesentlich förderte. Da kein eigentlicher Projektleiter vorhanden war, musste jedes Teammitglied über seinen Themenbereich hinaus mitdenken. Es entstand eine Lebhafte Diskussionskultur die viele gute Ansätze und Lösungen hervorbrachte. Da dies aber alles eher offen und frei vonstattenging, waren die Aufteilung und das Vorgehen danach nicht immer allen zu 100% klar. Aus diesem Grund wurden in PREN2 die sogenannten Planungssitzungen eingeführt. Mit diesem Instrument wurden die Besprechungen und die Beschlussfassung die sich während PREN1 ergeben hat, organisierter und vor allem Strukturierter. Die Planungssitzungen wurden jeweils alle zwei bis drei Wochen abgehalten. An diesen Sitzungen wurden die Hauptaufgaben für die nächste Periode besprochen und festgehalten. Es wurde definiert wer welche Arbeiten zu verrichten hat und welche Priorität diese haben. Weiter wurden Beschlüsse gefasst und Erkenntnisse und Auswirkungen der letzten Periode besprochen. Das ganze wurde jeweils in einem Protokoll festgehalten. (Verweis auf Anhang mit Protokollen) Im Anschluss an die Planungssitzungen wurde jeweils auch das Risikomangement besprochen und auf den neusten Stand gebracht. 
Die Projektplanung wurde nach demselben Muster wie in PREN1 geführt. Diese Excel Projektplanungsvorlage hat sich bewährt. Das Gantt-Diagramm ist einfach in der Handhabung und bietet eine grosse Übersichtlichkeit. Arbeiten welche im PREN1 abgeschlossen wurden, sind der Übersichthalben in dieser Planung nicht mehr aufgeführt. Allgemeine Projektarbeiten und Themengebiete welche sich über beide Semester erstrecken sind in die Planung von PREN2 übernommen worden. Um grösstmögliche Übersicht zu haben, ist die Projektplanung relativ allgemein gehalten. Das heisst es sind alle Themen und Arbeitsblöcke vorhanden, jedoch ist nicht jeder einzelne Arbeitsschritt der darunter anfällt, auch aufgeführt. Ebenso ist jeweils nur die verantwortliche Person aufgeführt. Sie trägt die Hauptverantwortung über ein Arbeitsblock, jedoch können auch andere Personen daran gearbeitet haben. Die Zeitangaben welche in Spalte drei enthalten sind, sind jeweils die Schätzungen die im Voraus vom Verantwortlichen des Projektteiles gemacht wurden. Genaue Angaben über geleistete Arbeitszeit wie auch ein Soll-Istzeit Vergleich sind unter (Verweis auf Kapitel) ersichtlich. Die Planung ist in einen Block allgemeine Projektarbeiten und einzelne Blöcke, welche die Disziplinen repräsentieren unterteilt.

    \subsection{Soll-/ Ist-Zeitvergleich}
\label{sec:SollIstVergleich}
\begin{zebratabular}{p{0.52\textwidth}p{0.1\textwidth}p{0.1\textwidth}p{0.15\textwidth}}
\rowcolor{gray}\multicolumn{4}{l}{\textbf{Allgemeine Projektarbeiten}}\\
\textbf{Aktivität} & \textbf{Planung} & \textbf{Ist} & \textbf{Abweichung}\\
Input              & 28  & 28   & -\\
Planungssitzungen  & 49  & 39.5 & -9.5\\
Dokumentation      & 200 &      & \\
Risikomanagement   & 10  & 5    & -5\\
Projektplanung     & 50  & 45   & -5\\
Plakat/Auftritt    & 10  &      & \\
Präsentation       & 30  &      & \\
\textbf{Gesamt}    & \textbf{377} &      & \\
                   &     &    & \\
\rowcolor{gray}\multicolumn{4}{l}{\textbf{Arbeiten Maschinentechnik}}\\
\textbf{Aktivität} & \textbf{Planung} & \textbf{Ist} & \textbf{Abweichung}\\
Komponenten Tests  & 25  & 35 & +10\\
Produktion         & 80  & 90 & +10\\
Montage            & 40  & 40 & -\\
Pläne/Zeichnungen  & 5   & 20 & +15\\
\textbf{Gesamt}    & \textbf{150} & \textbf{185} & \textbf{+35}\\
                   &     &    & \\
\rowcolor{gray}\multicolumn{4}{l}{\textbf{Arbeiten Elektrotechnik}}\\
\textbf{Aktivität}  & \textbf{Planung} & \textbf{Ist} & \textbf{Abweichung}\\
Schema              & 30 & 40  & +10\\
Print Prototyp      & 60 & 60  & -\\
Print definitiv     & 15 & 15  & -\\
Firmware            & 40 & 45  & +5\\
Stepper Board       & 12 & 8   & -4\\
DC Motor            & 3  & 8   & +5\\
Inbetriebnahme BLDC & 50 & 30  & -20\\
Inbetriebnahme      & 15 & 30  & +15\\
\textbf{Gesamt} & \textbf{225} & \textbf{236} & +\textbf{11}\\
                   &     &    & \\
\rowcolor{gray}\multicolumn{4}{l}{\textbf{Arbeiten Informatik}}\\
\textbf{Aktivität}          & \textbf{Planung} & \textbf{Ist} & \textbf{Abweichung}\\
Bluetooth-Connection        & 50 & 40 & -10\\
Desktop-Applikation         & 20 & 20 & -\\
Freedom-Board Kommunikation & 30 & 35 & +5\\
Android Zusatzkomponenten   & 12 & 15 & +3\\
Kamera-Komponente           & 12 & 8  & -4\\
Winkelberechnung            & 6  & 14 & +8\\
Mergen                      & 10 & 40 & +30\\
Integrationstest            & 30 & 30 & -\\
Wireless                    & 30 & 20 & -10\\
\textbf{Gesamt} & \textbf{200} & \textbf{222} & \textbf{+22}\\
                   &     &    & \\
\rowcolor{gray}\multicolumn{4}{l}{\textbf{Gemeinsame Arbeiten am Funktionsmuster}}\\
\textbf{Aktivität} & \textbf{Planung} & \textbf{Ist} & \textbf{Abweichung}\\
Optimierung/Parametereinstellung & 85  &  & \\
Puffer                           & 100 &  & \\
\textbf{Gesamt}                  &     &  & \\





\end{zebratabular} 

%\begin{table}[h!]
%    \begin{zebratabular}{p{0.10\textwidth}p{0.06\textwidth}p{0.25\textwidth}p{0.5\textwidth}}
%    \rowcolor{gray} Register & Wert & Beschreibung & Bemerkungen \\
%    TPM1SC &
%        \verb!0x0F! &
%        TPM1 Status and Control Register &
%        Overflow interrupt disabled, no Center-aligned PWM, Bus clock as clock 
%            source, Prescaler = 128\\
%    TPM1CNT &
%        \verb!0x____! &
%        TPM1 Counter Register &
%        No initialisation \\
%    TPM1MOD &
%        \verb!0x0000! &
%        TPM1 Counter Modulo Register &
%        Free running \\
%    TPM1C0SC &
%        \verb!0x50! &
%        TPM1 Channel 0 Status and Control Register &
%        Interrupt enabled, Output compare \\
%    TPM1C0V &
%        \verb!0x0000! &
%        TPM1 Channel 0 Value Register &
%        Used for commutation delay of phase U \\
%    TPM1C1SC &
%        \verb!0x50! &
%        TPM1 Channel 1 Status and Control Register &
%        Interrupt enabled, Output compare \\
%    TPM1C1V &
%        \verb!0x0000! &
%        TPM1 Channel 1 Value Register &
%        Used for commutation delay of phase V \\
%    TPM1C2SC &
%        \verb!0x50! &
%        TPM1 Channel 2 Status and Control Register &
%        Interrupt enabled, Output compare \\
%    TPM1C2V &
%        \verb!0x0000! &
%        TPM1 Channel 2 Value Register &
%        Used for commutation delay of phase W \\
%    TPM1C3SC &
%        \verb!0x44! &
%        TPM1 Channel 3 Status and Control Register &
%        Interrupt enable, input capture \\
%    TPM1C3V &
%        \verb!0x0000! &
%        TPM1 Channel 3 Value Register &
%        Initialized zero, value not used later \\
%    TPM1C4SC &
%        \verb!0x44! &
%        TPM1 Channel 4 Status and Control Register &
%        Interrupt enable, input capture \\
%    TPM1C4V &
%        \verb!0x0000! &
%        TPM1 Channel 4 Value Register &
%        Initialized zero, value not used later \\
%    TPM1C5SC &
%        \verb!0x44! &
%        TPM1 Channel 5 Status and Control Register &
%        Interrupt enable, input capture \\
%    TPM1C5V &
%        \verb!0x0000! &
%        TPM1 Channel 5 Value Register &
%        Initialized zero, value not used later \\
%    \end{zebratabular}
%    \caption{Registerinitialisierung TPM1}
%    \label{tab:rtc_init}
%\end{table} 
    \newpage
    \subsection{Kostenaufstellung}
In der Tabelle \ref{tab:KostenTabelleKurz} sind die gesamten Kosten für den autonomen 
Ballwerfer aufgeführt. Wie ersichtlich ist, konnte das in der Aufgabenstellung genannte 
Limit von SFR 600 eingehalten werden. Einige Bauteile sind Schenkungen von Firmen 
oder gehören einem Teammitglied. Die Kosten dieser Komponenten wurden mit Listenpreis 
oder als Schätzung in die Tabelle übernommen. Weiter standen für PREN2 noch die gesamten 
600 Franken zur Verfügung. Dies war möglich da die Versuche von PREN1 in vollem Umfang 
mit altem Material der Schule durchgeführt werden konnten. Ausserdem wurden viele Teile selbst 
hergestellt und nicht zugekauft. Eine komplette Liste sämtlicher verbauter Teile ist 
im Anhang \ref{apx:KostenaufstellungDetailliert} einsehbar.

	\begin{table}[h!]
		\begin{zebratabular}{p{0.05\textwidth}p{0.4\textwidth}p{0.1\textwidth}p{0.1\textwidth}}
			\rowcolor{gray}Stk. & Bezeichnung & Stückpreis & Gesamt\\
			\rule{0pt}{11pt}2  & Leitlager               &  Fr. 20.00  &  Fr. 40.00 \\
			\rule{0pt}{11pt}1  & Zahnrad Modul 0.5 - Z10 &  Fr.   9.17 &  Fr.  9.17 \\
			\rule{0pt}{11pt}3  & Zahnrad Modul 0.5 - Z50 &  Fr. 15.80  &  Fr. 47.40 \\
			\rule{0pt}{11pt}2  & Zahnrad Modul 1.25 Z30  &  Fr.   5.50 &  Fr. 11.00 \\
			\rule{0pt}{11pt}2  & Zahnrad Modul 1.25 Z15  &  Fr.   3.44 &  Fr.  6.88 \\
			\rule{0pt}{11pt}2  & Zahnrad Modul 1 Z30     &  Fr.   4.05 &  Fr.  8.10 \\
			\rule{0pt}{11pt}4  & Zahnrad Modul 1 Z15     &  Fr.   2.88 &  Fr. 11.52 \\
			\rule{0pt}{11pt}10 & Rillenkugellager        &  Fr.   1.39 &  Fr. 13.90 \\
			\rule{0pt}{11pt}1  & Schrittmotor            &  Fr. 71.25  &  Fr. 71.25 \\
			\rule{0pt}{11pt}2  & Brushlessmotor          &  Fr. 34.95  &  Fr. 69.90 \\
			\rule{0pt}{11pt}1  & Elektronikteile         &  Fr. 50.00  &  Fr. 50.00 \\
			\rule{0pt}{11pt}1  & Plexiglasmaterialien    &  Fr. 35.90  &  Fr. 35.90 \\ 
			\rule{0pt}{11pt}1  & Akku Smartphone         &  Fr.  19.65 &  Fr. 19.65  \\
			\rule{0pt}{11pt}1  & Förderband              &  Fr.  15.00 &  Fr. 15.00 \\
			\rule{0pt}{11pt}1  & Grip-Band               &  Fr.  10.00 &  Fr. 10.00 \\
			\rule{0pt}{11pt}1  & DC-Motor mit Getriebe   &  Fr.  20.00 &  Fr. 20.00 \\
			\rule{0pt}{11pt}1  & Tennisbälle             &  Fr. 22.90  &  Fr. 22.90 \\
			                   & \textbf{Total Kosten}   &             &  \textbf{Fr.462.57}  \\
		\end{zebratabular} 
	\centering
	\caption{Zusammenfassung der Kostentabelle}
	\label{tab:KostenTabelleKurz}
	\end{table}
    \newpage
    \input{Enddokumentation/Schlussdiskussion}
    \subsection{Lessons Learned}
Eine wichtige Lektion hat das Informatik-Team bezüglich Vorhandensein einer nützlichen, 
konkreten Fallbacklösung (Dank vorhandenem Risikomanagement) für Schlüssel Technologien gelernt. 
Da eine zufriedenstellende Kommunikation mittels Bluetooth nicht gewährleistet werden konnte, 
musste auf die Fallback-Variante, die WLAN als Kommunikationsmittel verwendet, zurückgegriffen werden. 
Das Refactoring des Codes beanspruchte, im Vergleich zur aufgewendet Zeit für das Lösen des Bluetooth-Problems, 
signifikant weniger Zeit. Das Informatik-Team hätte den Umstieg auf die Fallback-Lösung früher einleiten sollen, 
dass hätte einiges an wertvoller Entwicklungszeit eingespart.
\newline
\newline
Als Kommunikationsmittel zwischen dem Android-Phone und dem Desktop war anfangs PREN2 eine Verbindung 
mittels Bluetooth gedacht. Obwohl im PREN1 schon einzelne Test bezüglich verbinden der zwei Geräte und 
senden von Daten durchgeführt wurden, kam das Informatik-Team nicht umhin, das Konzept \enquote{Bluetooth} mit einem neuen 
Konzept zu ersetzen. Wie bereits in der Dokumentation des PREN1-Moduls beschrieben, war als Fallback-Lösung 
die Verwendung von WLAN gedacht. 
\\
\\
Die teamübergreifende Zusammenarbeit in der ET-Gruppe hat sich extrem bewährt.  Auf diese Weise 
konnte eine komplexere und anspruchsvolle Lösung realisiert werden. Dementsprechend war der 
Lerneffekt massiv grösser. Die ET-Zusammenarbeit hat sich nicht nur in der Erhöhung der man-power 
niedergeschlagen, sondern auch in der Vielfalt der Themen und deren spezifischen Problemen sowie 
Lösungen. Am Anfang war es zeitintensiv, die Gruppen, die Tools und das gemeinsame Vorgehen zu 
definieren und umzusetzen. Sobald dies erledigt war, funktionierte die Zusammenarbeit innerhalb 
der ET-Gruppe ausserordentlich gut.
    \newpage
    
    %Beginn Testberichte / Testprotokolle
    \section{Tests}
\subsection{Klebeversuch}
\begin{tabular}{p{3.6cm}p{\textwidth-3.6cm-0.7cm}}
    \rule{0pt}{11pt}\textit{Typ}              & Klebeversuch \\ 
    \rule{0pt}{11pt}\textit{Datum}:           & 06.03.2015   \\
    \rule{0pt}{11pt}\textit{Ort}:             & Teaminsel \\
    \rule{0pt}{11pt}\textit{Tester}:          & Matteo Trachsel \\
    \rule{0pt}{11pt}\textit{Ziel des Testes}: & Das Ziel dieses Testes bestand darin, den 
    gekauften Kleber UHU Hart auf seine Klebekraft und auf sein Erscheinungsbild zu testen. \\
    \rule{0pt}{11pt}\textit{Aufbau / Ablauf}: & 
    Für den Test werden verschiedene Acrylglas-Stücke zusammengeklebt.
    Hierfür wird der Kleber wie auf der Gebrauchsanweisung auf zwei Verfahren getestet. 
    Im ersten Versuch wird der UHU Kleber aufgetragen und die zwei Platten zusammengeklebt. 
    Im zweiten Versuch wird der Kleber zuerst auf die Acrylglasstücke aufgetragen und 
    gewartet bis er angetrocknet ist, danach noch einmal eine Schicht vom Kleber aufgetragen 
    und zusammengefügt.\\
    \rule{0pt}{11pt}\textit{Fazit / Verbesserungs-\newline vorschlag}: & 
    Mit dem Versuch konnte gezeigt werden, dass der Kleber sicher glasklar bleibt. Weiter 
    ist die erwünschte Klebekraft bestätigt worden. Beim zweiten Versuch, wo zuerst der 
    Kleber etwas angetrocknet wurde, ist eine deutlich schlechtere Klebekraft festgestellt 
    worden. Dadurch wird der Kleber immer sofort aufgeklebt.\\
\end{tabular}
    \section{Tests}
\subsection{Förderband}
\begin{tabular}{p{3.6cm}p{\textwidth-3.6cm-0.7cm}}
	\rule{0pt}{11pt}\textit{Typ}              & Förderband\\
	\rule{0pt}{11pt}\textit{Datum}:           & 13.03.2015   \\
	\rule{0pt}{11pt}\textit{Ort}:             & Werkstatt \\
	\rule{0pt}{11pt}\textit{Tester}:          & Pascal Roth und Matteo Trachsel  \\
	\rule{0pt}{11pt}\textit{Ziel des Testes}: & Das Ziel des Testes besteht darin, verschieden Führungsschaufeln zu testen und eine geeignete Befestigung zu finden. \\
	
	
	\rule{0pt}{11pt}\textit{Aufbau / Ablauf}: &
	Für den Test wurden aus einem 1 mm dicken Aluminiumblech, welches bereits auf die Breite des Förderbandes zugeschnitten wurde, verschieden Lange Stücke abgeschnitten. Da pro Tennisball zwei Führungsschaufeln vorne und hinten benötigt werden, gibt es zwei Möglichkeiten zur Gestaltung der Führungsschaufeln. Die erste Möglichkeit besteht darin, dass immer eine einzelne Schaufel für je vorne und hinten realisiert wird. Die zweite Möglichkeit besteht darin, dass man die hinter und die nächste vorne liegende Führungschaufel zusammen in einem Blechstück realisiert.\\
	
	
	\rule{0pt}{11pt}\textit{Fazit / Verbesserungs-\newline vorschlag}: &
	Durch den Versuch stellte sich heraus, dass es sich besser eignet, wenn zwei Führungsschaufeln zusammen in einem Blechstück realisiert werden. Wenn die Führungschaufelpaare mit dem UHU Kleber am vorderen Rand angeklebt werden, können sie immer noch den Radius der Wellen überfahren, ohne sich abzulösen. Zur Sicherheit können die Führungsschaufeln noch mit einem Klebeband befestigt werden. \\
\end{tabular}

\begin{figure}[h!]
	\includegraphics[width=0.9\textwidth,clip,trim=10cm 15cm 40cm 6cm]
	{Testberichte/Klebeversuch.jpg}
	\centering
	\caption{Förderband mit Führungschaufeln}
	\label{abb:Klebeversuch}
\end{figure}



    \subsection{Bildaufnahme mit Smartphone inklusive persistente Speicherung}
\begin{zebratabular}{p{4.5cm}p{\textwidth-3.6cm-0.7cm}}
    \rule{0pt}{11pt}\textit{Tester}              & Thomas Wiss \\ 
    \rule{0pt}{11pt}\textit{Datum}:           & 26.03.2015   \\
    \rule{0pt}{11pt}\textit{Ort}:             & Teaminsel \\
    \rule{0pt}{11pt}\textit{Beschreibung}:          & Mittels dem aufgenommenen Bild wird ein 
    Schwarz/Weiss-Abgleich erstellt und der PREN-Korb erkennt. Das Bild wird aufgenommen und 
    anschliessend in einem internen Verzeichnis des Smartphones abgespeichert. 
    Es handelt sich hier um einen Komponententest. \\
    \rule{0pt}{11pt}\textit{Akteure}:          & Operator zur Bedienung des Android-Phones. \\
    \rule{0pt}{11pt}\textit{Bedingung}:          & Lauffähiges Android-Phone mit 
    Android-Applikation \enquote{CameraSD}. \\
    \rule{0pt}{11pt}\textit{Erwartete Fehlermeldung}:          & keine \\
    \rule{0pt}{11pt}\textit{Vorgehen}:          & Applikation öffnen und Button \enquote{Take Picture}
     betätigen. Mittels Android-Explorer den Speicherort des Bildes verifizieren. \\
    \rule{0pt}{11pt}\textit{Erwartetes Ergebnis}:          & Gespeichertes Bild, am 
    vorgängig programmierten Speicherort. \\
    \rule{0pt}{11pt}\textit{Eingetretenes Ergebnis}:          & Bild in guter Qualität am 
    programmierten Speicherort gespeichert. \\
    \rule{0pt}{11pt}\textit{Test bestanden?}:          & Ja \\
    \rule{0pt}{11pt}\textit{Weiter Tests nötig?}:          & Nein \\
\end{zebratabular}    
   
    \subsection{Bluetooth-Kommunikation zwischen Smartphone und Desktop-PC}
\begin{zebratabular}{p{4.5cm}p{\textwidth-5.3cm}}
    \rule{0pt}{11pt}\textit{Tester}              & Livio Kunz \\ 
    \rule{0pt}{11pt}\textit{Datum}:           & 09.04.2015   \\
    \rule{0pt}{11pt}\textit{Ort}:             & Teaminsel \\
    \rule{0pt}{11pt}\textit{Beschreibung}:          & Via Bluetooth werden die Konfigurations- und Resultatsdaten (Items)zwischen dem Smartphone und dem Desktop-PC ausgetauscht. Mit diesem Test 
    verifiziert man das korrekte bidirektionale übermitteln, entpacken und verpacken der Items.	 \\
    \rule{0pt}{11pt}\textit{Akteure}:          & Operator zur Bedienung des Smartphones \\
    \rule{0pt}{11pt}\textit{Bedingung}:          & Lauffähiges Android-Phone mit 
    Android-Applikation. Desktop-PC mit lauffähiger Desktop-App  \\
    \rule{0pt}{11pt}\textit{Erwartete Fehlermeldung}:          & keine \\
    \rule{0pt}{11pt}\textit{Vorgehen}:          & Applikationen auf beiden Geräten öffnen, Bluetooth auf beiden Geräten einschalten. Verbindung von Smartphone auf Desktop starten, Pairing bestätigen\\
    \rule{0pt}{11pt}\textit{Erwartetes Ergebnis}:          & Serialisiertes Objekt wird empfangen und gelesen. \\
    \rule{0pt}{11pt}\textit{Eingetretenes Ergebnis}:          & Serialisiertes Objekt wurde gesendet und 
    erfolgreich deserialisiert. Alle Informationen enthalten. Bei mehrmaligem Senden trat Fehler auf. Objekt konnte nur einmal erfolgreich gesendet und empfangen werden!\\
    \rule{0pt}{11pt}\textit{Test bestanden?}:          & Nein \\
    \rule{0pt}{11pt}\textit{Weiter Tests nötig?}:          & Nein (Es wird auf Wireless(Websockets) umgestiegen) \\
\end{zebratabular}    
   

   
    \subsection{Wireless-Kommunikation zwischen Smartphone und Desktop-PC}
\begin{zebratabular}{p{4.5cm}p{\textwidth-5.3cm}}
    \rule{0pt}{11pt}\textit{Tester}              & Livio Kunz \\ 
    \rule{0pt}{11pt}\textit{Datum}:           & 30.04.2015   \\
    \rule{0pt}{11pt}\textit{Ort}:             & Teaminsel \\
    \rule{0pt}{11pt}\textit{Beschreibung}:          & Um die Konfigurations- und Resultatsdaten (Items) 
    zwischen dem Smartphone und dem Desktop-PC auszutauschen, verwenden wir Sockets. Mit diesem Test 
    verifiziert man das korrekte bidirektionale übermitteln, entpacken und verpacken der Items.	 \\
    \rule{0pt}{11pt}\textit{Akteure}:          & Operator zur Bedienung des Smartphones \\
    \rule{0pt}{11pt}\textit{Bedingung}:          & Lauffähiges Android-Phone mit 
    Android-Applikation. Desktop-PC mit lauffähiger Desktop-App  \\
    \rule{0pt}{11pt}\textit{Erwartete Fehlermeldung}:          & keine \\
    \rule{0pt}{11pt}\textit{Vorgehen}:          & Applikationen auf beiden Geräten öffnen, Wireless beim 
    Android-Phone einschalten. Verbindung von Desktop Seite starten \\
    \rule{0pt}{11pt}\textit{Erwartetes Ergebnis}:          & Serialisiertes Objekt wird empfangen und gelesen. Kann mehrmals gesendet/empfangen werden. \\
    \rule{0pt}{11pt}\textit{Eingetretenes Ergebnis}:          & Serialisiertes Objekt wurde gesendet und 
    erfolgreich deserialisiert. Alle Informationen enthalten.\\
    \rule{0pt}{11pt}\textit{Test bestanden?}:          & Ja \\
    \rule{0pt}{11pt}\textit{Weiter Tests nötig?}:          & Nein \\
\end{zebratabular}    
   

   
    \newpage
    \subsection{Förderband Ballzuführung}
\begin{zebratabular}{p{4.5cm}p{\textwidth-5.3cm}}
    \rule{0pt}{11pt}\textit{Tester}           & Matteo Trachsel\\ 
    \rule{0pt}{11pt}\textit{Datum}:           & 26.03.2015   \\
    \rule{0pt}{11pt}\textit{Beschreibung}:    & Um die Bälle möglichst schnell zu den Beschleunigungsräder zu fördern, muss das Förderband gleichmässig, zentral und ohne Schlupf auf der Welle laufen. \\
    \rule{0pt}{11pt}\textit{Akteure}:         & Welle mit DC-Motor für Antrieb und Stirnrädern, Förderriemen mit aufgeklebten Leitschaufeln, Achse mit Spannelement. \\
    \rule{0pt}{11pt}\textit{Bedingung}:       & Funktionsmuster soweit montiert dass Förderreimen betrieben werden kann.\\
    \rule{0pt}{11pt}\textit{Erwartete Fehlermeldung}:          & keine \\
    \rule{0pt}{11pt}\textit{Vorgehen}:        & Förderreimen einsetzen, spannen, mit fünf Bällen bestücken. Spannung an DC-Motor anlegen. \\
    \rule{0pt}{11pt}\textit{Erwartetes Ergebnis}: & Bälle werden nach vorne befördert. Förderreimen bleibt mittig auf Achse und Welle bombiert. Kein Schlupf zwischen Welle und Riemen. \\
    \rule{0pt}{11pt}\textit{Eingetretenes Ergebnis}: & Funktionsfähigkeit bestätigt. Erkenntnis: Schlupf auf Welle ist kein Problem. Je weniger der Riemen gespannt ist, desto gleichmässiger ist der Lauf. Keine Gefahr des seitlichen Ablaufens des Riemens. \newline
    Negativ: Durch ungenaue Bearbeitung läuft Stirnrad nicht Rund. 
    \\
    \rule{0pt}{11pt}\textit{Test bestanden?}:     & Ja, allerdings neues Stirnrad bestellen und nochmals bearbeiten.  \\
    \rule{0pt}{11pt}\textit{Weiter Tests nötig?}: & Nein \\
\end{zebratabular}  

\begin{figure}[h!]
	\includegraphics[width=0.5\textwidth,clip,trim=0cm 40cm 0cm 35cm]
	{Testberichte/Foerderband.jpg}
	\centering
	\caption{Förderband mit Führungschaufeln}
	\label{abb:Förderband mit Führungschaufeln}
\end{figure}
    \subsection{Winkelverstellung}
\begin{zebratabular}{p{4.5cm}p{\textwidth-5.3cm}}
    \rule{0pt}{11pt}\textit{Tester}           & Roger Gisler\\ 
    \rule{0pt}{11pt}\textit{Datum}:           & 17.04.2015\\
    \rule{0pt}{11pt}\textit{Beschreibung}:    & Um den Ballwerfer zum Korb hin auszurichten, ist eine Verstelleinheit montiert. Diese verändert den Winkel des Ballwerfers gegenüber seiner Anfangsposition bis auf einen gewünschten Wert. \\
    \rule{0pt}{11pt}\textit{Akteure}:         & Schrittmotor mit Ritzel welches in den Zahnkranz am Ende der Grundplatte eingreift.\\
    \rule{0pt}{11pt}\textit{Bedingung}:       & Funktionsmuster soweit montierte. Bälle auf Förderband. Mit zusätzlichen Gewichten belastet um Endgewicht von zu simulieren. \\
    \rule{0pt}{11pt}\textit{Erwartete Fehlermeldung}:          & keine \\
    \rule{0pt}{11pt}\textit{Vorgehen}:        & Schrittmotor ansteuern, und in beide Drehrichtungen drehen. Ineinandergreifen der Verzahnung, ruckfreier Lauf, Festigkeit der Bauteile überprüfen. \\
    \rule{0pt}{11pt}\textit{Erwartetes Ergebnis}: & Ruckfreier Lauf, optimaler Eingriff der Verzahnung über den gesamten Verstellwinkel gewährleistet. Verstellung in beide Richtungen. \\
    \rule{0pt}{11pt}\textit{Eingetretenes Ergebnis}: & Ritzel und Zahnrad greifen über den gesamten Verstellwinkel optimal ineinander. Kraft des Schrittmotor ausreichend. 
    Festigkeit des Ritzels aus Acrylglas nicht ausreichend, was zum Bruch des Bauteils führte. 
    Stabilität des gesamten Ballwerfers kritisch da sich ein grosser Teil des Gewichts auf dem vorderen Drehpunkt befindet. \\
    \rule{0pt}{11pt}\textit{Test bestanden?}:     & Nein \\
    \rule{0pt}{11pt}\textit{Weiter Tests nötig?}: & Ja \\
    \rule{0pt}{11pt}\textit{Weiteres Vorgehen}: & Ritzel aus MDF (höhere Elastizität) Lasern.
    Vorderer Quersteg(Acrylglas) auf dem ein grosser Teil des Gewichts lastet aus Aluminium fertigen um die Durchbiegung und Schwingungen zu eliminieren. 
    Grundplatte im vorderen Bereich rund um den Drehpunkt neu Konstruieren um die Kippgefahr zu eliminieren und die Seitliche Stabilität zu erhöhen. \\
\end{zebratabular}  
    \subsection{Zahnradgetriebe der Beschleunigungsräder}
\begin{zebratabular}{p{4.5cm}p{\textwidth-5.3cm}}
    \rule{0pt}{11pt}\textit{Tester}           & Pascal Roth\\ 
    \rule{0pt}{11pt}\textit{Datum}:           & 09.04.2015\\
    \rule{0pt}{11pt}\textit{Beschreibung}:    & Um die Kraft der Motoren auf die Beschleunigungsräder zu übertragen, muss Eingriff und Wellenabstand der Zahnräder stimmen.\\
    \rule{0pt}{11pt}\textit{Akteure}:         & Je zwei Zahnradpaarungen pro Beschleunigungsrad.\\
    \rule{0pt}{11pt}\textit{Bedingung}:       & Achsen, Wellen und Zahnräder des Getriebes montiert, Achsabstand auf variabler Seite eingestellt. Motoren montiert (nicht angeschlossen)\\
    \rule{0pt}{11pt}\textit{Erwartete Fehlermeldung}:          & keine \\
    \rule{0pt}{11pt}\textit{Vorgehen}:        & Motor manuell drehen. Eingriff, Rundlauf, axiale Verschiebung der Zahnräder überprüfen. \\
    \rule{0pt}{11pt}\textit{Erwartetes Ergebnis}: & Eingriff, Achsabstand, Rundlauf stimmen analog CAD Modell\\
    \rule{0pt}{11pt}\textit{Eingetretenes Ergebnis}: & Alles IO.\\
    \rule{0pt}{11pt}\textit{Test bestanden?}:     & Ja \\
    \rule{0pt}{11pt}\textit{Weiteres Vorgehen}: & Test mit erhöhter Drehzahl\\
\end{zebratabular}  
    \subsection{Zahnradgetriebe der Beschleunigungsräder II}
\begin{zebratabular}{p{4.5cm}p{\textwidth-3.6cm-0.7cm}}
    \rule{0pt}{11pt}\textit{Tester}           & Pascal Roth / Yves Studer\\ 
    \rule{0pt}{11pt}\textit{Datum}:           & 16.04.2015\\
    \rule{0pt}{11pt}\textit{Beschreibung}:    & Erweiterung des ersten Zahnradgetriebe Tests. Antrieb durch Motoren mit Nenndrehzahl.\\
    \rule{0pt}{11pt}\textit{Akteure}:         & Je zwei Zahnradpaarungen pro Beschleunigungsrad. Brushless-Motoren inkl. deren Ansteuerung.\\
    \rule{0pt}{11pt}\textit{Bedingung}:       & Achsen, Wellen und Zahnräder des Getriebes montiert, Achsabstand auf variabler Seite eingestellt. Motoren montiert, angeschlossen und funktionsfähig.\\
    \rule{0pt}{11pt}\textit{Erwartete Fehlermeldung}:          & keine \\
    \rule{0pt}{11pt}\textit{Vorgehen}:        & Motoren auf Nenndrehzahl beschleunigen. Verhalten des Getriebes bezüglich Schwingungen und Vibrationen überprüfen. Festigkeit der Welle-Nabe Verbingdungen.\\
    \rule{0pt}{11pt}\textit{Erwartetes Ergebnis}: & Verhalten des ersten Testlaufs auch mit höherer Drehzahl bestätigt. \\
    \rule{0pt}{11pt}\textit{Eingetretenes Ergebnis}: & Eingriff der Verzahnung auch mit hohen Drehzahlen okay. Keine übermässigen Vibrationen. Welle-Nabe Verbindung Zahnrad – Welle der Beschleunigungsräder ist dem Beschleunigungsmoment nicht gewachsen. Achse aus Gleitlagermaterial iglidur ist der Thermischen Belastung durch die Reibung nicht gewachsen.
    \\
    \rule{0pt}{11pt}\textit{Test bestanden?}:     & Verzahnung Ja\newline
    Lagerung Nein\newline
    Welle-Nabe Nein\\
    \rule{0pt}{11pt}\textit{Weiteres Tests nötig}: & Ja\\
    \rule{0pt}{11pt}\textit{Weiteres Vorgehen}: & Defekte Achse aus Aluminium fertigen. Phase in Welle der Beschleunigungsräder Fräsen damit Stellschraube mehr Halt findet. \\
\end{zebratabular}  
    \subsection{Zahnradgetriebe der Beschleunigungsräder III}
\begin{zebratabular}{p{4.5cm}p{\textwidth-5.3cm}}
    \rule{0pt}{11pt}\textit{Tester}           & Pascal Roth / Yves Studer\\ 
    \rule{0pt}{11pt}\textit{Datum}:           & 23.04.2015\\
    \rule{0pt}{11pt}\textit{Beschreibung}:    & Wiederholung des Tests Zahnradgetriebe II mit neuer Achse und optimierter Welle-Nabe Verbindung. \\
    \rule{0pt}{11pt}\textit{Akteure}:         & Je zwei Zahnradpaarungen pro Beschleunigungsrad. Brushless-Motoren inkl. deren Ansteuerung. Speziell Achse des Getriebes und Welle-Nabe Verbindung zwischen der Welle der Beschleunigungsräder und dessen Zahnrad.\\
    \rule{0pt}{11pt}\textit{Bedingung}:       & Achsen, Wellen und Zahnräder des Getriebes montiert, Achsabstand auf variabler Seite eingestellt. Motoren montiert, angeschlossen und funktionsfähig.\\
    \rule{0pt}{11pt}\textit{Erwartete Fehlermeldung}:          & keine \\
    \rule{0pt}{11pt}\textit{Vorgehen}:        & Motoren auf Nenndrehzahl beschleunigen. Sitz der erwähnten Welle-Nabe Verbindung unter verschiedenen Betriebsbedingungen und Lastwechsel. Verhalten der Gleitlagerpaarung Alu-Achse Kunststoffzahnrad.\\
    \rule{0pt}{11pt}\textit{Erwartetes Ergebnis}: & Bauteile sind nun den erhöhten Belastungen gewachsen. \\
    \rule{0pt}{11pt}\textit{Eingetretenes Ergebnis}: & Welle-Nabe Verbindung auch nach mehreren Beschleunigungszyklen und Lastwechsel noch fest. Neue Achse aus Aluminium ist der Belastung gewachsen. 
    \\
    \rule{0pt}{11pt}\textit{Test bestanden?}:     & Ja\\
    \rule{0pt}{11pt}\textit{Weiteres Vorgehen}: & Nein\\
\end{zebratabular}  
    \subsection{Stabilität des mechanischen Aufbaus}
\begin{zebratabular}{p{4.5cm}p{\textwidth-3.6cm-0.7cm}}
    \rule{0pt}{11pt}\textit{Tester}           & Roger Gisler\\ 
    \rule{0pt}{11pt}\textit{Datum}:           & 21.05.2015\\
    \rule{0pt}{11pt}\textit{Beschreibung}:    & Fortsetzung Test \enquote{Winkelverstellung}
    Stabilität des Gesamten Ballwerfers mit den überarbeiteten Teilen überprüfen. \\
    \rule{0pt}{11pt}\textit{Akteure}:         & Komplettes Funktionsmuster \\
    \rule{0pt}{11pt}\textit{Bedingung}:       & Funktionsmuster komplett montiert und mit Bällen bestückt.\\
    \rule{0pt}{11pt}\textit{Erwartete Fehlermeldung}:          & keine \\
    \rule{0pt}{11pt}\textit{Vorgehen}:        & Stabilität, Schwingverhalten, Vibrationen am Ballwerfer unter verschiedensten Lastbedingungen testen.\\
    \rule{0pt}{11pt}\textit{Erwartetes Ergebnis}: & Durchbiegung des vorderen Querstegs eliminiert. Stark verringerte seitliche Schwingungen.\\
    \rule{0pt}{11pt}\textit{Eingetretenes Ergebnis}: & Keine Durchbiegung mehr der vorderen Strebe. Dadurch keine vertikalen Schwingungen mehr. Seitliche Bewegungen oder Schwingungen sind durch die breitere Auflagefläche fast komplett verschwunden.\\
    \rule{0pt}{11pt}\textit{Test bestanden?}:     & Ja\\
    \rule{0pt}{11pt}\textit{Weiteres Tests nötig}: & Nein\\
\end{zebratabular}  
    \clearpage
    \subsection{Kommunikation zwischen Freedom-Board und Android-Phone}
\begin{zebratabular}{p{4.5cm}p{\textwidth-5.3cm}}
    \rule{0pt}{11pt}\textit{Tester}              & Thomas Wiss \\ 
    \rule{0pt}{11pt}\textit{Datum}:           & 16.04.2015   \\
    \rule{0pt}{11pt}\textit{Ort}:             & Teaminsel \\
    \rule{0pt}{11pt}\textit{Beschreibung}:          & Um den auf dem Android-Phone berechneten 
    Winkel zum Freedom-Board zu senden, muss die Kommunikation zwischen dem Freedom-Board 
    und dem Android-Phone getestet werden. \\
    \rule{0pt}{11pt}\textit{Akteure}:          & Operator zur Bedienung des Android-Phones. \\
    \rule{0pt}{11pt}\textit{Bedingung}:          & USB-Kabel muss am Android-Phone und gleichzeitig 
    auch am Freedom-Board eingesteckt sein. \\
    \rule{0pt}{11pt}\textit{Erwartete Fehlermeldung}:          & keine \\
    \rule{0pt}{11pt}\textit{Vorgehen}:          & USB-Kabel am Android-Phone und am 
    Freedom-Board einstecken. Starten der Applikation \enquote{SerialPortExample} auf dem Android-Phone. 
    Den Befehl \enquote{debug} ins Textfeld eintragen und den Button Send betätigen. \\
    \rule{0pt}{11pt}\textit{Erwartetes Ergebnis}:          & Die Rückmeldung in der TextBox 
    muss ein Informations- und Hilfemenu anzeigen. \\
    \rule{0pt}{11pt}\textit{Eingetretenes Ergebnis}:          & Siehe Abbildung 
    \ref{abb:ScreenshotSerialPortExampleTestProtokol}. \\
    \rule{0pt}{11pt}\textit{Test bestanden?}:          & Ja \\
    \rule{0pt}{11pt}\textit{Weiter Tests nötig?}:          & Nein \\
\end{zebratabular}    
    \newline
    \newline
    \begin{figure}[h!]
    	\includegraphics[width=0.3\textwidth,clip,trim=0mm 0mm 0mm 0mm]
    	{Enddokumentation/Bilder/Screenshot_SerialPortExample_debug.png}
    	\centering
    	\caption{Screenshot der Applikation \enquote{SerialPortExample}}
    	\label{abb:ScreenshotSerialPortExampleTestProtokol}
    \end{figure}
    

    \subsection{Korberkennung}
\begin{zebratabular}{p{4.5cm}p{\textwidth-5.3cm}}
    \rule{0pt}{11pt}\textit{Tester}              & Niklaus Manser, Livio Kunz, Thomas Wiss \\ 
    \rule{0pt}{11pt}\textit{Datum}:           & 17.05.2015   \\
    \rule{0pt}{11pt}\textit{Ort}:             & Teaminsel \\
    \rule{0pt}{11pt}\textit{Beschreibung}:          & Um die Funktionalität der Korberkennung zu belegen, soll der Detektor in der Lage sein, den Korb in jeder Möglichen Position auf dem Spielfeld zu erkennen.	 \\
    \rule{0pt}{11pt}\textit{Akteure}:          & Operator zur Bedienung der Desktop-App \\
    \rule{0pt}{11pt}\textit{Bedingung}:          & Lauffähiges Android-Phone mit 
    Android-Applikation. Desktop-PC mit lauffähiger Desktop-App  \\
    \rule{0pt}{11pt}\textit{Erwartete Fehlermeldung}:          & keine \\
    \rule{0pt}{11pt}\textit{Vorgehen}:          & Applikationen auf beiden Geräten öffnen, Wireless beim 
    Android-Phone einschalten. Verbindung von Desktop Seite starten. Korb in verschiedenen Positionen aufstellen und den Erkennungsmechanismus starten. \\
    \rule{0pt}{11pt}\textit{Erwartetes Ergebnis}:          & Korb wird erkannt und Position im Bild bestimmt. \\
    \rule{0pt}{11pt}\textit{Eingetretenes Ergebnis}:          & Der Korb konnte in allen gewählten Positionen gefunden werden. \\
    \rule{0pt}{11pt}\textit{Test bestanden?}:          & Ja \\
    \rule{0pt}{11pt}\textit{Weiter Tests nötig?}:          & Ja, Korberkennung soll bis zum Wettbewerb kontinuierlich weiter getestet werden. \\
\end{zebratabular}    
   

   
    \clearpage
    % Testprotkolle Informatik als PDF's
    \newpage
    
    
    % % % %
    \begin{flushleft}
        \setlength\bibitemsep{2\itemsep}
        \nocite{*} %Alle Quellen ausgeben
        \renewcommand{\refname}{Literatur- und Quellenverzeichnis}
%        \{\refname}{Quellenverzeichnis}
        \bibliography{Enddokumentation/ET-Gruppe/et-gruppe_source,../common/Quellen}{} %!!! Kein Leerzeichen nach dem , !!!!
    \end{flushleft}
    %
    % Beginn des Anhangs
    %
    \appendix      
   	\begin{appendix}
   		\clearpage
   		\pagenumbering{Roman} % römische Nummerierung des Anhangs (Grosse Buchstaben)
   		\section{Anhang}
   		
   		\section{Rapporte}
   		
   		\subsection{Woche 1 (20.02.2015)}
\label{sec:BeginnRaporte}
\textbf{Hauptaufgaben in den folgenden zwei Wochen.}
\begin{itemize}
    \item Planung PREN2 erstellen.
    \item Teile bestellen/in Produktion geben M
    \item ET-Tools installieren und initialisieren
    \item Schema beginnen
    \item Bestellungen auslösen E
    \item Grobplanung Informatik erstellen
\end{itemize}
\begin{table}[h!]
    \begin{zebratabular}{p{0.6\textwidth}p{0.1\textwidth}p{0.1\textwidth}p{0.1\textwidth}}
        \rowcolor{gray} Aufgabe & Wer & Priorität & Erledigt \\
        Grundplanung PREN2 erstellen           & Pascal & 1 & x\\
        Informatik Grobplanung erstellen       & I      & 1 & x\\
        Scrum Initalisierung                   & I      & 2 & x\\
        Plexi Lasern dxf                       & M      & 1 & x\\
        Drehteile in Produktion geben          & M      & 1 & x\\
        Normteile, Rohmaterial bestellen       & M      & 1 & x\\
        USB Connection Android - FreedomBoard  & I      & 1 & x\\
        Schema erstellen                       & Yves   & 1 & \\
          &       &  & \\
          &       &  & \\
          &       &  & \\
          &       &  & \\
          &       &  & \\
          &       &  & \\
    \end{zebratabular}
\end{table}
Priorität: 1 (hoch) - 4 (tief)\\
\textbf{Wichtige Beschlüsse}
\begin{itemize}
    \item Informatik-Abteilung zusätzliche Planung mit Scrum
    \item Risikomanagement jeweils nach Planungssitzung besprechen und aktualisieren
    \item 
\end{itemize}
\begin{table}[h!]
    \begin{zebratabular}{p{0.1\textwidth}p{0.85\textwidth}}
        \rowcolor{gray} Wer & Erkenntnis \\
         & Keine da Anfang PREN2\\
         & \\
         & \\
         & \\
         & \\
    \end{zebratabular}
\end{table}
   		\newpage
   		\subsection{Woche 3 (12.03.2015)}
\textbf{Hauptaufgaben in den folgenden zwei Wochen.}
\begin{itemize}
    \item Zusammenbau Funktionsmuster soweit wie möglich
    \item Hardware
    \item 
    \item 
    \item 
    \item 
\end{itemize}
\begin{table}[h!]
    \begin{zebratabular}{p{0.6\textwidth}p{0.1\textwidth}p{0.1\textwidth}p{0.1\textwidth}}
        \rowcolor{gray} Aufgabe & Wer & Priorität & Erledigt \\
        Unpassende Laserteile neu in Auftrag geben              & M      & 2 & x\\
        Bestellung der Zahnräder kontrollieren, evtl. bestellen & M      & 1 & x\\
        Platine bestücken                                       & Yves   & 1 & x\\
        Software zur Platine schreiben                          & Yves   & 1 & x\\
        Kamera Android implementieren                           & Thomas & 1 & x\\
        Bluetooth Seriaziable Object senden/empfangen           & Livio  & 1 & \\
        Desktop App GUI                                         & Nik    & 3 & x\\
        Projektteile mergen (Kamera/Bluetooth/USB)              & I      & 3 & \\
    \end{zebratabular}
\end{table}
Priorität: 1 (hoch) - 4 (tief)\\
\textbf{Wichtige Beschlüsse}
\begin{itemize}
    \item Solidworks installieren um CAD Modell von Manuel anpassen zu können.
    \item 
    \item 
\end{itemize}
\begin{table}[h!]
    \begin{zebratabular}{p{0.1\textwidth}p{0.85\textwidth}}
        \rowcolor{gray} Wer & Erkenntnis \\
         Alle & Zeichnungen/Listen/Bestellungen/Berechnungen gegenkontrollieren!\\
         M    & Wareneingang, Teile auf Richtigkeit überprüfen\\
         I    & Scrum für dieses Projekt unnötig/ zu viel overhead!\\
         I    & FTDI unnötig, da Freedom-Board direkt angesprochen werden kann\\
         & \\
    \end{zebratabular}
\end{table}
   		\newpage
   		\subsection{Woche 5 (20.03.2015)}
\textbf{Hauptaufgaben in den folgenden zwei Wochen.}
\begin{itemize}
    \item Fehlende Komponenten bestellen und montieren
    \item Schrittmotor Abklärungen und bestellen
    \item Produzieren/löten von Print
    \item 
    \item 
    \item 
\end{itemize}
\begin{table}[h!]
    \begin{zebratabular}{p{0.6\textwidth}p{0.1\textwidth}p{0.1\textwidth}p{0.1\textwidth}}
        \rowcolor{gray} Aufgabe & Wer & Priorität & Erledigt \\
        Führung für Beschleunigungsräder anpassen      & M      & 1 & x\\
        Verstrebungen neu Lasern                       & M      & 2 & x\\
        Motorbefestigung                               & M      & 1 & x\\
        Zahnräder, Förderband                          & M      & 2 & x\\
        Gleitlager                                     & M      & 1 & x\\
        Grundplatte lasern (mit Manu absprechen)       & M      & 2 & x\\
        Schrittmotor                                   & M      & 2 & x\\
        Grundplatte (Manuel)                           & M      & 2 & x\\
        Bluetooth Seriaziable Object senden/empfangen  & Livio  & 1 & \\
        Desktop App GUI                                & Nik    & 2 & x\\
        Projektteile mergen (Kamera/Bluetooth/USB)     & I      & 2 & x\\
        Print in Produktion geben                      & Yves   & 1 & x\\
        Print löten                                    & I      & 1 & x\\
    \end{zebratabular}
\end{table}
Priorität: 1 (hoch) - 4 (tief)\\
\textbf{Wichtige Beschlüsse}
\begin{itemize}
    \item Alte Android Kamera Version wird verwendet
    \item Interface zwischen Freedom-Board und Android Phone festgelegt
    \item 
\end{itemize}
\begin{table}[h!]
    \begin{zebratabular}{p{0.1\textwidth}p{0.85\textwidth}}
        \rowcolor{gray} Wer & Erkenntnis \\
         M   & Günstige oder einfache Teile mehrfach herstellen oder bestellen, um bei allfälligen Defekten nicht von Lieferfristen abhängig zu sein. \\
             & \\
             & \\
             & \\
         & \\
    \end{zebratabular}
\end{table}
   		\newpage
   		\subsection{Woche 7 (09.04.2015)}
\textbf{Hauptaufgaben in den folgenden zwei Wochen.}
\begin{itemize}
    \item Erste Tests einzelner Komponenten
    \item Software Integrationstests
    \item Inbetriebnahme Komponenten
    \item 
    \item 
    \item 
\end{itemize}
\begin{table}[h!]
    \begin{zebratabular}{p{0.6\textwidth}p{0.1\textwidth}p{0.1\textwidth}p{0.1\textwidth}}
        \rowcolor{gray} Aufgabe & Wer & Priorität & Erledigt \\
        Schrittmotor                                   & Matteo & 1 & x\\
        Firmware                                       & Yves   & 1 & x\\
        Stepperboard anpassen/Inbetriebnahme           & Yves   & 1 & x\\
        Bluetooth Seriaziable Object senden/empfangen  & Livio  & 1 & \\
        Integrationstest                               & I      & 1 & \\
        Testen des Zahnradgetriebes                    & Roger  & 1 & x\\
        Motoren definitiv montieren                    & M      & 1 & x\\
        Produktion Grundplatte                         & Matteo & 1 & x\\
        1.Inbetriebnahme BLDC Board                    & Yves   & 1 & \\
        Testen der Winkelverstellung                   & Roger  & 2 & x\\
              &       &  & \\
              &       &  & \\
              &       &  & \\
    \end{zebratabular}
\end{table}
Priorität: 1 (hoch) - 4 (tief)\\
\textbf{Wichtige Beschlüsse}
\begin{itemize}
    \item Änderung der Motorbefestigung, da die vorgesehene Methode nicht möglich ist. (Motor Aussenläufer)
    \item 
    \item 
\end{itemize}
\begin{table}[h!]
    \begin{zebratabular}{p{0.1\textwidth}p{0.85\textwidth}}
        \rowcolor{gray} Wer & Erkenntnis \\
         M   & Allfällige Abweichungen vom CAD Modell und Maschinentoleranzen berücksichtigen.\\
             & \\
             & \\
             & \\
         & \\
    \end{zebratabular}
\end{table}
   		\newpage
   		\subsection{Woche 9 (24.04.2015)}
\textbf{Hauptaufgaben in den folgenden zwei Wochen.}
\begin{itemize}
    \item Inbetriebnahme
    \item Verbesserung der Mechanischen Komponenten
    \item Dokumentation Schreiben
    \item 
    \item 
    \item 
\end{itemize}
\begin{table}[h!]
    \begin{zebratabular}{p{0.6\textwidth}p{0.1\textwidth}p{0.1\textwidth}p{0.1\textwidth}}
        \rowcolor{gray} Aufgabe & Wer & Priorität & Erledigt \\
        Wireless implementieren                 & Livio & 1 & x\\
        Erweiterung Freedom-b. Kommunikation    & Thomas& 1 & x\\
        (Div. Verbesserungen nach Tests)        & M     &   & \\
        Beschreiben der Komponenten für Doku    & M     & 2 & \\
        Komplette Teileliste erstellen          & Roger & 2 & x\\
        Inbetriebnahme Gesamtsystem             & Yves  & 1 & x\\
        Einstellung des Reglers                 & Yves  & 2 & x\\
              &       &  & \\
              &       &  & \\
              &       &  & \\
              &       &  & \\
              &       &  & \\
              &       &  & \\
    \end{zebratabular}
\end{table}
Priorität: 1 (hoch) - 4 (tief)\\
\textbf{Wichtige Beschlüsse}
\begin{itemize}
    \item Datenübertragung Bluetooth nicht möglich wie geplant. Wird mit W-LAN realisiert. 
    \item Konsolenbefehle definitiv abgesprochen
    \item 
\end{itemize}
\begin{table}[h!]
    \begin{zebratabular}{p{0.1\textwidth}p{0.85\textwidth}}
        \rowcolor{gray} Wer & Erkenntnis \\
         Alle& Von nun an ist es wichtig, dass jeweils alle anwesend sind!\\
         I   & Planung einer Fallback-Variante hat sich als sehr hilfreich herausgestellt.\\
             & \\
             & \\
         & \\
    \end{zebratabular}
\end{table}
   		\newpage
  		\subsection{Woche 11 (08.05.2015)}
\textbf{Hauptaufgaben in den folgenden zwei Wochen.}
\begin{itemize}
    \item Testen, Testen, Testen
    \item Verbesserungen aufgrund Testresultate
    \item Dokumentation möglichst fertigstellen
    \item 
    \item 
    \item 
\end{itemize}
\begin{table}[h!]
    \begin{zebratabular}{p{0.6\textwidth}p{0.1\textwidth}p{0.1\textwidth}p{0.1\textwidth}}
        \rowcolor{gray} Aufgabe & Wer & Priorität & Erledigt \\
        Logo, Plakat, T-Shirt            & Pascal & 2 & x\\
        Software Deployment              & I      & 1 & x\\
        Dokumentation                    & I      & 1 & x\\
        Dokumentation                    & M      & 1 & x\\
        Dokumentation                    & Yves   & 1 & x\\
        Wireless integration             & Livio  & 1 & x\\
        Kamerahalter montieren           & Matteo & 1 & x\\
              &       &  & \\
              &       &  & \\
              &       &  & \\
              &       &  & \\
              &       &  & \\
              &       &  & \\
              &       &  & \\
    \end{zebratabular}
\end{table}
Priorität: 1 (hoch) - 4 (tief)\\
\textbf{Wichtige Beschlüsse}
\begin{itemize}
    \item Es müssen auch Wochenenden und Abende zum Testen reserviert werden. 
    \item 
    \item 
\end{itemize}
\begin{table}[h!]
    \begin{zebratabular}{p{0.1\textwidth}p{0.85\textwidth}}
        \rowcolor{gray} Wer & Erkenntnis \\
         I   & Interface Freedom-Board zu Phone überarbeiten\\
         M   & Gleitlager müssen gereinigt und geschmiert werden nach einigen Tests\\
         M   & Festsitz der Schrauben vor jeder Testrunde überprüfen\\
             & \\
         & \\
    \end{zebratabular}
\end{table}
  		\newpage

  
		\includepdf[landscape=true,page=1 , offset=0cm -1.5cm, width=1.65\textwidth,picturecommand={\centering},pagecommand=\section{Kostenaufstellung Detailliert}{\thispagestyle{fancy}},]{Enddokumentation/Anhang/Kostentabelle.pdf}
		
		\includepdf[landscape=true,page=2-3 , offset=0cm -.5cm, width=1.8\textwidth,picturecommand={\centering},pagecommand={\thispagestyle{fancy}},]{Enddokumentation/Anhang/Kostentabelle.pdf}
   		
   	\end{appendix}  
\end{document}